\documentclass[10pt]{book}
\usepackage[width=5.5in,height=8.5in,hmarginratio=3:2,vmarginratio=1:1]{geometry}
\usepackage[utf8]{inputenc}
\usepackage[russian]{babel}
\usepackage{graphicx}
\graphicspath{{assets/}{../assets/}}
\usepackage{hyperref}
\usepackage{subfiles}
\usepackage{glossaries}
%\usepackage{listings}
\makeglossaries

\begin{document}

    \pagestyle{empty}   
    \title{Динамические формы: Исчерпывающее руководство для новичков}
	\author{Кирилл Кривошеев \thanks{hard-won with blood and sweat by the happy team ДРУГ}}
    \date{18 Апреля 2019}
    
    \begin{titlepage}
    \maketitle
    \end{titlepage}

    \mainmatter
    \tableofcontents

	\chapter{Введение}
	\subfile{sections/introduction}
	
	% Структура программы =====================================================================
	\chapter{Структура программы}
    \subfile{sections/structure}

	% Динамическая форма =====================================================================
	\chapter{Динамическая форма}\label{sec:dynfom} 
    \subfile{sections/dynform}
	% Группы ==================================================================================
    \subfile{sections/groups}


	% Асинхронность ===========================================================================
    \chapter{Асинхронные команды}
    \subfile{sections/async}

	% Паттерны ================================================================================
	\chapter{Паттерны программирования на динамическом языке}
    \subfile{sections/patterns}

    % Глоссарий ================================================================================
    \clearpage
    \printglossaries 

\end{document}
