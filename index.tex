\documentclass[14pt]{extbook}
\usepackage[width=165mm, height=235mm, left=25mm, right=25mm, top=30mm, bottom=30mm]{geometry}
\usepackage[utf8]{inputenc}
\usepackage[russian]{babel}
\usepackage{graphicx}
\graphicspath{{assets/}{../assets/}}
\usepackage{hyperref}
\usepackage{subfiles}
\usepackage{glossaries}
\usepackage{booktabs}
\usepackage{epigraph}
\usepackage{fancyhdr}
\usepackage{nopageno}
\usepackage{indentfirst}
\setlength{\parindent}{1cm}

\makeglossaries

\begin{document}
		
 	\pagestyle{empty} 
    \title{Динамические формы: Исчерпывающее руководство для новичков}
	\author{Кирилл Кривошеев \thanks{hard-won with blood and sweat by the happy team ДРУГ}}
    \date{18 Апреля 2019}
    
    \begin{titlepage}
    \maketitle
    \end{titlepage}

    \mainmatter
    \tableofcontents
	
	\pagestyle{fancy} 
	\fancyhead[RE]{\leftmark}
	\fancyhead[LE]{\thepage}
	\fancyhead[LO]{\rightmark}
	\fancyhead[RO]{\thepage}
	\fancyfoot[]{}

	\setcounter{secnumdepth}{0}

    \chapter{История создания динамических форм}
    \subfile{sections/prolog}
    
    \setcounter{secnumdepth}{3}
    
	\chapter{Введение}
    \subfile{sections/introduction}
   	
	% Структура программы =====================================================================
	\chapter{Структура программы}
    \subfile{sections/structure}

	% Динамическая форма =====================================================================
	\chapter{Динамическая форма}\label{sec:dynfom}
	Основа пользовательского интерфейса, его разметка, стили, поведение и будущий результат работы пользователя,
	были воплощены в набор стандартизированных данных, называемый «Динамическая форма». Она свободно интерпретируется
	специализированными приложениями для web-браузеров (портал «Лицо ДРУГа») и приложениями для операционных систем:
	iOS и Android. В этой главе мы рассмотрим структуру и правила описания и контроля динамической формы
	на всех этапах её жизненного цикла.
	
    \subfile{sections/dynform}
    
    \subfile{sections/groups}

	% Паттерны ================================================================================
	\chapter{Паттерны программирования на динамическом языке}
    \subfile{sections/patterns}

    % Справочник
    \appendix
    \setcounter{secnumdepth}{0}
    
    \chapter{Описание динамических команд}\label{apx:dlib_doc}
    \subfile{sections/dlib_doc}

    \chapter{Примеры 3DMap структуры}\label{apx:3dmap}
    \subfile{sections/apx3dmap.tex}
%    \chapter{Справочник компонентов дин форм}\label{apx:components}

    % Глоссарий ================================================================================
    \clearpage
    \printglossaries 

\end{document}
