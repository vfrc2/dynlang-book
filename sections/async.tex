\documentclass[../index.tex]{subfiles}
\begin{document}
\section{Автоисполнятор}
Автоисполнятор или АИ -- представляет собой Task Manager или систему асинхронного выполнения задач. Задачи представляют собой скрипты написанные на языке groovy. Groovy -- объектно-ориентированный язык программирования разработанный для платформы Java как альтернатива языку Java с возможностями Python, Ruby и Smalltalk. Каждый скрипт идентифицируется по уникальному имени (например: \verb|FRIEND_GET_LIMIT_BY_USER_TPL_DOUBLE|. 
Кроме имени скрипт на вход принимает дополнительные параметры. 
Существует два типа скриптов - стандартные и скрипты с обработкой динамической формы
Стандартные скрипты на вход принимают только данные заданные непосредственно в функции динамического языка. Функции с обработкой формы дополнительно преобразовывают текущее состояние формы в JSON объект и передает ее в тело функции.
Для вызова АИ в динамическом языке используют функции: \textbf{functionName}  и \textbf{\_functionName}, где \textbf{functionName} является любым именем функции, которая не входим в список динамических функций(см. \autoref{sec:dlib_doc})
Функции без нижнего подчеркивания делают запрос к стандартным скриптам, с нижнем подчеркиванием -- к скриптам с обработкой формы.
Количество аргументов таких функций не ограничено. Первым аргументом передается название скрипта, а последним -- элемент над которым производится выполнение.

\section{Сопрограммы}

\end{document}