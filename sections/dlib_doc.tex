\providecommand{\tightlist}{%
	\setlength{\itemsep}{0pt}\setlength{\parskip}{0pt}}

\setlength{\LTleft}{0pt}

\pagestyle{empty}

\hypertarget{setvaluevalue-id-string}{%
\section{setValue(value, id)}\label{setvaluevalue-id-string}}

Устанавливает значение первого параметра в элемент, указанный во втором
параметре.

\begin{enumerate}
\def\labelenumi{\arabic{enumi}.}
\tightlist
\item
  Если элемент не select и значение первого параметра строка, то в text
  запишется строкоговое значение
\item
  Если элемент не select и значение первого параметра Map, то в text
  запишется первое значение из Map
\item
  Если элемент не select и значение первого параметра пустая Map-а, то в
  text запишется пустая строка
\item
  Если элемент select и первый параметр Map, то options будут записаны
  значения из Map
\item
  Если элемент seleсt и первый параметр пустая Map, то будут удалены все
  options и в text будет устанвлена пустая строка.
\item
  Если элемент select и значение первого параметра строка, то будет
  добавлен один option c label и text равный значение второго параметра.
\end{enumerate}

Возвращает: \protect\hyperlink{string}{string} - Значение второго
параметра (идентификатор элемента)

\begin{longtable}[]{@{}lll@{}}
\toprule
Параметр & Тип & Описание\tabularnewline
\midrule
\endhead
value & \protect\hyperlink{string}{string} \textbar{} DLMap & Новое
значение элемента\tabularnewline
id & \protect\hyperlink{string}{string} & Идентификатор
элемента\tabularnewline
\bottomrule
\end{longtable}

\textbf{Пример}

\begin{verbatim}
Элемент:

<form>
<text id="1" label="Наименование товара" style="text" type="2">
Ручка шариковая для сотрудника
</text>
</form>

1. setValue([Значение 1],[this]) -
 установит в текущий элемент значение test

Результат:

<form>
  <text id="1"
 label="Наименование товара" 
 style="text" type="2">
 Значение 1
   </text>
</form>

2. setValue([DLMap{"1": "Значение 1", "2": "Значение 2"}], [this])

Результат:

<form>
  <text id="1" _parentId="0" 
label="Наименование товара" style="text" type="2">
Значение 1
  </text>
</form>

3. setValue([DLMap{}], [this])

Результат:

<form>
<text id="1"  label="Наименование товара" style="text" type="2"/>
</form>

Элемент:

<form>
<select id="f1"> Выбранное значение
  <option id="f1_0" 
label="Отсутствует оплата за замещение должностей">
Выбранное значение</option>
  </select>
</form>

4. setValue([DLMap{"1": "Значение 1", "2": "Значение 2"}], [this])

Результат:

<form>
<select id="f1"> Выбранное значение
<option id="1" label="Значение 1">1</option>
<option id="2" label="Значение 2">2</option>
</select>
</form>

5. setValue([DLMap{}], [this])

Результат:

<form>
<select id="f1">
</select>
</form>

6.setValue([Значение 1],[this])

Результат:

<form>
<select id="f1"> Выбранное значение
<option id="Значение 1" label="Значение 1">Значение 1</option>
</select>
</form>
\end{verbatim}

\hypertarget{getuserattr-string-boolean}{%
\section{getUser(attr)}\label{getuserattr-string-boolean}}

Функция возвращает информацию о пользователе

Возвращает: \protect\hyperlink{string}{string} \textbar{} boolean
- Информация о пользователе

\begin{longtable}[]{@{}lll@{}}
\toprule
Параметр & Тип & Описание\tabularnewline
\midrule
\endhead
attr & \protect\hyperlink{string}{string} & Значение
агрумента\tabularnewline
\bottomrule
\end{longtable}

\begin{verbatim}
Допустимые значения агрумента:

getuser(fio)
getuser(firstName)
getuser(secondName)
getuser(middleName)
getuser(tabNum)
getuser(departName)
getuser(position) - Должность
getuser(vsp)
getuser(address)
getuser(phone)
getuser(phoneInner)
getuser(phoneMobile)
getuser(identificator)
getuser(id)
getuser(kadrCode)
getuser(category)
getuser(terbank)
getuser(findByTabno)
getuser(territoryCode)
\end{verbatim}

\hypertarget{runargs-boolean}{%
\section{run(args)}\label{runargs-boolean}}

Выполняет вложенные выражения. Порядок выполнения выражений не
определен. Функция доступна только в выражениях динамического языка.

Возвращает: boolean - Возвращает всегда ``true''

\begin{longtable}[]{@{}ll@{}}
\toprule
Параметр & Описание\tabularnewline
\midrule
\endhead
args & Список выражений динамического языка\tabularnewline
\bottomrule
\end{longtable}

\textbf{Пример}

\begin{verbatim}
run(
  [setValue([1],[id1])],
  [setValue([2],[id2])],
  [setValue([3],[id3])
])
\end{verbatim}

\hypertarget{ifcondition-trueexpr-falseexpr-boolean}{%
\section{if(condition, trueExpr, falseExpr)}\label{ifcondition-trueexpr-falseexpr-boolean}}

Вычисляет выражение в первом аргументе, если результат вычислений равно
true то выполняется выражение во втором аргументе, иначе третий

Возвращает: boolean - Возвращает значение первого аргумента
true/false

\begin{longtable}[]{@{}ll@{}}
\toprule
Параметр & Описание\tabularnewline
\midrule
\endhead
condition & Условие\tabularnewline
trueExpr & Выражение 1\tabularnewline
falseExpr & Выражение 2\tabularnewline
\bottomrule
\end{longtable}

\textbf{Пример}

\begin{verbatim}
if([idDigit([3])],
  [setValue([Число],[this])],
  [setValue([Строка],
[this])])
\end{verbatim}

\hypertarget{isinargs-boolean}{%
\section{isIn(args)}\label{isinargs-boolean}}

Провереряет первый аргумент равен одному из следующих аргументов


\begin{longtable}[]{@{}lll@{}}
\toprule
Параметр & Тип & Описание\tabularnewline
\midrule
\endhead
args & DLMap \textbar{} * & Список параметров\tabularnewline
\bottomrule
\end{longtable}

\textbf{Пример}

\begin{verbatim}
isIn([1],[2],[56],[1])
\end{verbatim}

\hypertarget{isnotinargs-boolean}{%
\section{isNotIn(args)}\label{isnotinargs-boolean}}

Провереряет первый аргумент не равен одному из следующих аргументов


\begin{longtable}[]{@{}ll@{}}
\toprule
Параметр & Описание\tabularnewline
\midrule
\endhead
args & Список параметров\tabularnewline
\bottomrule
\end{longtable}

\textbf{Пример}

\begin{verbatim}
isNotIn([1],[2],[56],[1])
\end{verbatim}

\hypertarget{isdigitarg-boolean}{%
\section{isDigit(arg)}\label{isdigitarg-boolean}}

Проверяет, является ли заданный аргумент числом

\begin{longtable}[]{@{}ll@{}}
\toprule
Параметр & Описание\tabularnewline
\midrule
\endhead
arg & Аргумент\tabularnewline
\bottomrule
\end{longtable}

\textbf{Пример}

\begin{verbatim}
isDigit([34])
\end{verbatim}

\hypertarget{callexpr-id-string}{%
\section{call(expr, id)}\label{callexpr-id-string}}

Выполняет команду динамического языка, переданную в первом параметре над
элементом второго параметра, поддерживает выражения с асинхронными
вызовами.

Возвращает: \protect\hyperlink{string}{string} идентификатор второго аргумента

\begin{longtable}[]{@{}lll@{}}
\toprule
Параметр & Тип & Описание\tabularnewline
\midrule
\endhead
expr & & Выражение на динамическом языке\tabularnewline
id & \protect\hyperlink{string}{string} & Идентификатор
элемента\tabularnewline
\bottomrule
\end{longtable}

\textbf{Пример}

\begin{verbatim}
call([setValue([test],[this])],[id])
\end{verbatim}

\hypertarget{iamarg-boolean}{%
\section{IAm(arg)}\label{iamarg-boolean}}

Функция проверяет является ли пользователь приложения заявителем,
иницатором или согласующим лицом в текущем обращении


\begin{longtable}{@{} *{1}{l} p{10cm} @{}}
\toprule
Параметр & Описание\tabularnewline
\midrule
\endhead
arg & Параметр может принимать значения creator, \newline  initiator или
 approver\tabularnewline
\bottomrule
\end{longtable}

\textbf{Пример}

\begin{verbatim}
iAm([creator])
\end{verbatim}

\hypertarget{gettextsmap-dlmap}{%
\section{getTexts(map)}\label{gettextsmap-dlmap}}

Возвращает Map которая в value содержит видимые значения ``label''
option-элементов.

Возвращает: DLMap Map которая в value содержит
видимые значения ``label'' option-элементов.

\begin{longtable}[]{@{}lll@{}}
\toprule
Параметр & Тип & Описание\tabularnewline
\midrule
\endhead
map & DLMap & Map-а ключи которой содержат id элементов на
форме\tabularnewline
\bottomrule
\end{longtable}

Наример, если в getTexts передать Map \verb|{"f1": "-"}|c id элемента "f1" и выполнить над формой:

\begin{verbatim}
<form>
<select id="f1"> Значение 1
  <option id="f1_0"
 label="Отсутствует оплата за замещение должностей">
 Значение 1
   </option>
</select>
</form>
\end{verbatim}

Функция вернет Map структуру вида
\begin{verbatim}
"f1\_Отсутствует оплата за замещение должностей" :
"Отсутствует оплата за замещение должностей"}
\end{verbatim}

\hypertarget{getvaluesmap-dlmap}{%
\section{getValues(map) }\label{getvaluesmap-dlmap}}

Возвращает Map которая в value содержит значения ``text''
option-элементов.

Возвращает: DLMap - Возвращает Map которая в value содержит
значения ``text'' option элементов.

\begin{longtable}[]{@{}lll@{}}
\toprule
Параметр & Тип & Описание\tabularnewline
\midrule
\endhead
map & DLMap & Map ключи которой содержат id элементов на
форме\tabularnewline
\bottomrule
\end{longtable}

\textbf{Пример}

Если в getValues передать Map \verb|{"f1": "-"}| c id элемента "f1" и выполнить над формой

\begin{verbatim}
<form>
  <select id="f1"> Значение 1
    <option id="f1_0"
label="Отсутствует оплата за замещение должностей">Значение 1
    </option>
  </select>
</form>
\end{verbatim}

Функция вернет Map структуру
\begin{verbatim}
{
"f1_Отсутствует оплата за замещение должностей" :
"Значение 1"
}
\end{verbatim}

\hypertarget{findattr-value-group-dlmap}{%
\section{find(attr, value, group)}\label{findattr-value-group-dlmap}}

Универсальная функция, выполняет поиск элементов в зависимисти от
первого аргумента. Если первый аргумент равен ``id'' то выполняется
поиск элементов, id которых начинается со второго аргумента. Если первый
аргумент равен ``type'', то выполняется поиск элементов, sbtype - тип
которых равен второму аргументу. В третьем параметре можно передать id
группы в которой нужно выполнить поиск.

Возвращает: DLMap - Map в которой ключи это id найденных
элементов, а value их значения\\

\begin{longtable}[]{@{}lll@{}}
\toprule
\begin{minipage}[b]{0.30\columnwidth}\raggedright
Param\strut
\end{minipage} & \begin{minipage}[b]{0.30\columnwidth}\raggedright
Type\strut
\end{minipage} & \begin{minipage}[b]{0.30\columnwidth}\raggedright
Description\strut
\end{minipage}\tabularnewline
\midrule
\endhead
\begin{minipage}[t]{0.30\columnwidth}\raggedright
attr\strut
\end{minipage} & \begin{minipage}[t]{0.30\columnwidth}\raggedright
\protect\hyperlink{string}{string}\strut
\end{minipage} & \begin{minipage}[t]{0.30\columnwidth}\raggedright
Атрбут по которому будет выполнен поиск, принимает значения ``id'' или
``type''\strut
\end{minipage}\tabularnewline
\begin{minipage}[t]{0.30\columnwidth}\raggedright
value\strut
\end{minipage} & \begin{minipage}[t]{0.30\columnwidth}\raggedright
\protect\hyperlink{string}{string}\strut
\end{minipage} & \begin{minipage}[t]{0.30\columnwidth}\raggedright
Значение для поиска\strut
\end{minipage}\tabularnewline
\begin{minipage}[t]{0.30\columnwidth}\raggedright
group\strut
\end{minipage} & \begin{minipage}[t]{0.30\columnwidth}\raggedright
\protect\hyperlink{string}{string}\strut
\end{minipage} & \begin{minipage}[t]{0.30\columnwidth}\raggedright
Идентификатор группы в которой будет выполнен поиск\strut
\end{minipage}\tabularnewline
\bottomrule
\end{longtable}

\textbf{Пример}

\begin{verbatim}
Пример формы:

<form>
  <text id="f1" _parentId="0" 
label="Наименование товара" style="text" type="2">
Ручка шариковая для сотрудника</text>

  <select id="f2"
 label="Вопрос" sbtype="suggest"  style="combo">
    <option id="id0empty" label=""/>
    <option id="id00" 
label="Отсутствует оплата за работу в ночное время">a1</option>
    <option id="id01" 
label="Отсутствует оплата за замещение должностей">a2</option>
  </select>
</form>

1. find([id], [f])

Результат:

DLMap {"f1": "Ручка шариковая для сотрудника","f2": ""}

2. find([type],[suggest])

Результат:

DLMap {"f2": ""}
\end{verbatim}

\hypertarget{plusargs}{%
\section{plus(args)}\label{plusargs}}

Универсальная функция сложения, которая умеет складывать числа, строки и
время в формате HH:mm Первый аргумент может использоваться
для формата чисел (для форматирования используется реализация
\href{https://docs.oracle.com/javase/7/docs/api/java/text/DecimalFormat.html}{Java
DecimalFormat})

Возвращает: Результат сложения

\begin{longtable}[]{@{}ll@{}}
\toprule
Параметр & Описание\tabularnewline
\midrule
\endhead
args & Список аргументов\tabularnewline
\bottomrule
\end{longtable}

\textbf{Пример}

\begin{verbatim}
plus([1],[2],[3])
6
plus([],[],[1],[],[2],[3],[],[4],[],[])
10
plus([1],[3],[a])
13a
plus([1],[01:02],[3])
66
plus([a],[01:02],[3])
a01:023
plus([#0.00],[123],[0,0000001])
123.00
plus([1],[DLMap{a : 1, b: 2, c: 3}],[3])
10
plus([1],[DLMap{a : 01:00, b: 2, c: 3}],[3])
69
plus([a],[DLMap{a : 01:00, b: 2, c: 3}],[3])
a01:00233
\end{verbatim}

\hypertarget{minusargs}{%
\section{minus(args)}\label{minusargs}}

Функция вычитания Первый аргумент может использоваться для формата чисел
(для форматирования используется JDK
DecimalFormat

Возвращает: * - Результат вычитания из первого аргумента всех
последующих аргументов

\begin{longtable}[]{@{}ll@{}}
\toprule
Параметр & Описание\tabularnewline
\midrule
\endhead
args & Числа, например {[}1{]},{[}2{]},{[}3{]}\tabularnewline
\bottomrule
\end{longtable}

\textbf{Пример}

\begin{verbatim}
minus([4],[2],[1])
1
minus([#0.00],[123,0000002],[0,0000001])
123.00
\end{verbatim}

\hypertarget{mulargs}{%
\section{mul(args)}\label{mulargs}}

Функция умножения Первый аргумент может использоваться для формата чисел
(для форматирования используется JDK
DecimalFormat

Возвращает: * - Результат умножения

\begin{longtable}[]{@{}ll@{}}
\toprule
Параметр & Описание\tabularnewline
\midrule
\endhead
args & Числа, например {[}1{]},{[}2{]},{[}3{]}\tabularnewline
\bottomrule
\end{longtable}

\hypertarget{divargs}{%
\section{div(args)}\label{divargs}}

Функция деления Первый аргумент может использоваться для формата чисел
(для форматирования используется JDK
DecimalFormat

Возвращает: * - Результат деления

\begin{longtable}[]{@{}ll@{}}
\toprule
Параметр & Описание\tabularnewline
\midrule
\endhead
args & Числа, например {[}1{]},{[}2{]},{[}3{]}\tabularnewline
\bottomrule
\end{longtable}

\hypertarget{indofstring-substring-number}{%
\section{indOf(string, substring)}\label{indofstring-substring-number}}

Возвращает позицию с которой начинается подстрока substring в строке
source

Возвращает: number \textbar{} * - Позиция в строке

\begin{longtable}[]{@{}lll@{}}
\toprule
Параметр & Тип & Описание\tabularnewline
\midrule
\endhead
string & DLMap \textbar{} \protect\hyperlink{string}{string} &
Строка\tabularnewline
substring & DLMap \textbar{} \protect\hyperlink{string}{string} &
Искомая подстрока\tabularnewline
\bottomrule
\end{longtable}

\textbf{Пример}

\begin{verbatim}
indOf([console.log],[ole])
4
\end{verbatim}

\hypertarget{substrstring-start-end-string}{%
\section{substr(string, start, end)}\label{substrstring-start-end-string}}

Возвращает подстроку с позиции start до позиции end

Возвращает: \protect\hyperlink{string}{string} - Подстрока

\begin{longtable}[]{@{}lll@{}}
\toprule
Параметр & Тип & Описание\tabularnewline
\midrule
\endhead
string & \protect\hyperlink{string}{string} \textbar{} DLMap & Строка в
которой произоводится поиск\tabularnewline
start & number \textbar{} * & Начальная позиция\tabularnewline
end & number \textbar{} * & Конечная позиция (не включая)\tabularnewline
\bottomrule
\end{longtable}

\textbf{Пример}

\begin{verbatim}
substr([console.log], [2], [4])
\end{verbatim}

\hypertarget{lenstring-number}{%
\section{len(string)}\label{lenstring-number}}

Возвращает длину строки

Возвращает: number \textbar{} * - Длина строки

\begin{longtable}[]{@{}lll@{}}
\toprule
Параметр & Тип & Описание\tabularnewline
\midrule
\endhead
string & DLMap \textbar{} \protect\hyperlink{string}{string} &
Строка\tabularnewline
\bottomrule
\end{longtable}

\textbf{Пример}

\begin{verbatim}
len([console.log])
11
\end{verbatim}

\hypertarget{copygroupgroupid-id-group-string}{%
\section{copyGroup(groupId, id, group)}\label{copygroupgroupid-id-group-string}}

Копирует группу элементов. Поддерживает два вида групп: группировка по
атрибуту setGroupId (ПОДРУГА) или sbgroup (SMIT); группировка по типу
элемента - (object type) group. Возвращает второй аргумент.

Возвращает: \protect\hyperlink{string}{string} - Значение второго
аргумента

\begin{longtable}[]{@{}lll@{}}
\toprule
\begin{minipage}[b]{0.30\columnwidth}\raggedright
Param\strut
\end{minipage} & \begin{minipage}[b]{0.30\columnwidth}\raggedright
Type\strut
\end{minipage} & \begin{minipage}[b]{0.30\columnwidth}\raggedright
Description\strut
\end{minipage}\tabularnewline
\midrule
\endhead
\begin{minipage}[t]{0.30\columnwidth}\raggedright
groupId\strut
\end{minipage} & \begin{minipage}[t]{0.30\columnwidth}\raggedright
\protect\hyperlink{string}{string}\strut
\end{minipage} & \begin{minipage}[t]{0.30\columnwidth}\raggedright
Идентификатор группы\strut
\end{minipage}\tabularnewline
\begin{minipage}[t]{0.30\columnwidth}\raggedright
id\strut
\end{minipage} & \begin{minipage}[t]{0.30\columnwidth}\raggedright
\protect\hyperlink{string}{string}\strut
\end{minipage} & \begin{minipage}[t]{0.30\columnwidth}\raggedright
Идентификатор\strut
\end{minipage}\tabularnewline
\begin{minipage}[t]{0.30\columnwidth}\raggedright
group\strut
\end{minipage} & \begin{minipage}[t]{0.30\columnwidth}\raggedright
\protect\hyperlink{string}{string}\strut
\end{minipage} & \begin{minipage}[t]{0.30\columnwidth}\raggedright
Идентификтор группы в которой нужно выполнить копирование группы\strut
\end{minipage}\tabularnewline
\bottomrule
\end{longtable}

\hypertarget{getdistanceargs-dlmap}{%
\section{getDistance(args)}\label{getdistanceargs-dlmap}}

Возвращает из Яндекса расстояние между координатами

Возвращает: DLMap - Словарь со значениями distance - расстояние в
метрах, distance\_text - расстояних в км., time - время поездки в
секундах, time\_text - время в часах

\begin{longtable}[]{@{}lll@{}}
\toprule
Параметр & Тип & Описание\tabularnewline
\midrule
\endhead
args & DLMap & Cловарь состоящий из координат
``широта:долгота''\tabularnewline
\bottomrule
\end{longtable}

\textbf{Пример}

\begin{verbatim}
address =   { // "longitude:latitude:GUID"
                 "a":"37.589283:55.745850:GUID"},
             "b":"36.715736:55.745850:GUID",
             "c":"37.515736:55.673816:GUID",
             "d":"37.715736:55.773816:GUID",
           }

getDistance([address])

Резульат:

DLMap {
 distance: '353637.28597939014',
 distance_text: '350 км',
 time: '25415.397077530622',
 time_text: '7 ч 4 мин'
 }
\end{verbatim}

\hypertarget{execute3dtasktask-arg0-arg1-argn-map-dlmap}{%
\section{execute3DTask(TASK, args, map)}\label{execute3dtasktask-arg0-arg1-argn-map-dlmap}}

Получение данных из автосполнятора в виде 3DMap

Возвращает: DLMap - Ответ от автосполнятора

\begin{longtable}[]{@{}lll@{}}
\toprule
Параметр & Тип & Описание\tabularnewline
\midrule
\endhead
TASK & DLMap \textbar{} \protect\hyperlink{string}{string} & Название
задачи в автосполняторе\tabularnewline
ARG0 & DLMap \textbar{} \protect\hyperlink{string}{string} &
Произвольный аргумет 1\tabularnewline
ARG1 & DLMap \textbar{} \protect\hyperlink{string}{string} &
Произвольный аргумет 2\tabularnewline
ARGN & DLMap \textbar{} \protect\hyperlink{string}{string} &
Произвольный аргумет N\tabularnewline
map & DLMap & Произвольный аргумет\tabularnewline
\bottomrule
\end{longtable}

\hypertarget{executetasktask-arg0-arg1-argn-object}{%
\section{executeTask}\label{executetasktask-arg0-arg1-argn-object}}

Получение данных из автосполнятора

Возвращает: Object \textbar{} * - Ответ от автосполнятора

\begin{longtable}[]{@{}lll@{}}
\toprule
Параметр & Тип & Описание\tabularnewline
\midrule
\endhead
TASK & DLMap \textbar{} \protect\hyperlink{string}{string} & Название
задачи в автосполняторе\tabularnewline
ARG0 & DLMap \textbar{} \protect\hyperlink{string}{string} &
Произвольный аргумет\tabularnewline
ARG1 & DLMap \textbar{} \protect\hyperlink{string}{string} &
Произвольный аргумет\tabularnewline
ARGN & DLMap \textbar{} \protect\hyperlink{string}{string} &
Произвольный аргумет\tabularnewline
\bottomrule
\end{longtable}

\section{execute3DTask
\label{execute3dtasktask-arg0-arg1-argn-dlmap-object}}
Получение данных из автосполнятора в виде 3DMap объекта с actions

Возвращает: DLMap \textbar{} * - Ответ от автосполнятора

\begin{longtable}[]{@{}lll@{}}
\toprule
Параметр & Тип & Описание\tabularnewline
\midrule
\endhead
TASK & DLMap \textbar{} \protect\hyperlink{string}{string} & Название
задачи в автосполняторе\tabularnewline
ARG0 & DLMap \textbar{} \protect\hyperlink{string}{string} &
Произвольный аргумет\tabularnewline
ARG1 & DLMap \textbar{} \protect\hyperlink{string}{string} &
Произвольный аргумет\tabularnewline
ARGN & DLMap \textbar{} \protect\hyperlink{string}{string} &
Произвольный аргумет\tabularnewline
DLMap & DLMap \textbar{} \protect\hyperlink{string}{string} &
текущее состояние динамической формы.\tabularnewline
\bottomrule
\end{longtable}

\hypertarget{getusersbyorganizationrequestrealtyobj-text-object}{%
\section{getUsersByOrganizationRequest}\label{getusersbyorganizationrequestrealtyobj-text-object}}

Получение пользователя по подразделению


\begin{longtable}[]{@{}lll@{}}
\toprule
Параметр & Тип & Описание\tabularnewline
\midrule
\endhead
realtyObj & DLMap \textbar{} \protect\hyperlink{string}{string} &
Идентификатор обьекта недвижимости\tabularnewline
text & DLMap \textbar{} \protect\hyperlink{string}{string} & Поисковый
запрос\tabularnewline
\bottomrule
\end{longtable}

\hypertarget{setvaluesarg0-arg1-arg2-arg3-arg4}{%
\section{setValues(arg0-arg4)}\label{setvaluesarg0-arg1-arg2-arg3-arg4}}

Выполняет комманды 3DMaps над динамичской формой. Поддерживаются
команды: attr, value, new, valueselected, command, delete, beforesave

Возвращает: null id - элемент над которым выполняет 3DMap(может
отсутствовать, если запрос к АИ) tid - идентификатор запроса который
выполняется в рамках обработки callback - функция обработки
isDefaultSettings - включает или отключает silentMode\\

\begin{longtable}[]{@{}lll@{}}
\toprule
Параметр & Тип & Описание\tabularnewline
\midrule
\endhead
ARG0 & 3DMap & map 3DMap структруа с коммандами\tabularnewline
ARG1 & \protect\hyperlink{string}{string} & id or tid\tabularnewline
ARG2 & \protect\hyperlink{string}{string} \textbar{} function & tid or
callback\tabularnewline
ARG3 & function \textbar{} undefined & callback\tabularnewline
ARG4 & boolean \textbar{} undefined & isDefaultSettings\tabularnewline
\bottomrule
\end{longtable}

\textbf{Пример}

\begin{verbatim}
Пример 3DMap.
[
 {
   "id": "cost",
   "actions": [
     {
       "action": "value",
       "values": [
         {
           "id": "22.25",
           "value": "22.25"
         }
       ]
     },
     {
       "action": "command",
       "values": [
         {
           "id": "100",
           "value": "setSaveEnabled([true],[cost])"
         }
       ]
     },
     {
       "action": "attr",
       "values": [
         {
           "id": "error",
           "value": ""
         },
         {
           "id": "rightValue",
           "value": "true"
         }
       ]
     }
   ]
 }
 ]
// map, id, tid, callback, isDefaultSettings
\end{verbatim}

\hypertarget{getvalueid}{%
\section{getValue(id)}\label{getvalueid}}

Возвращает текущее значение элемента

Возвращает: DLMap со значениями элемента или пустую

\begin{longtable}[]{@{}ll@{}}
\toprule
Параметр & Описание\tabularnewline
\midrule
\endhead
id & id элемента\tabularnewline
\bottomrule
\end{longtable}

\hypertarget{setmandatorybool-id}{%
\section{setMandatory(bool, id)}\label{setmandatorybool-id}}

Функция устанавливает обязательность заполнения контрола

Возвращает: id контрола

\begin{longtable}[]{@{}lll@{}}
\toprule
Параметр & Тип & Описание\tabularnewline
\midrule
\endhead
bool & &\tabularnewline
id & String \textbar{} DLMap & id контола\tabularnewline
\bottomrule
\end{longtable}

\hypertarget{setvisiblebool-id}{%
\section{setVisible(bool, id)}\label{setvisiblebool-id}}

Функция устанавливает признак видимости элемента на форме

Возвращает: id контола

\begin{longtable}[]{@{}lll@{}}
\toprule
Параметр & Тип & Описание\tabularnewline
\midrule
\endhead
bool & &\tabularnewline
id & String \textbar{} DLMap & id контола\tabularnewline
\bottomrule
\end{longtable}

\hypertarget{setenabledbool-id}{%
\section{setEnabled(bool, id)}\label{setenabledbool-id}}

Функция устанавливает признак возможности редактирования элемента

Возвращает: id контола

\begin{longtable}[]{@{}lll@{}}
\toprule
Параметр & Тип & Описание\tabularnewline
\midrule
\endhead
bool & &\tabularnewline
id & String \textbar{} DLMap & id контола\tabularnewline
\bottomrule
\end{longtable}

\hypertarget{setvalidbool-description-id}{%
\section{setValid(bool, description,
id)}\label{setvalidbool-description-id}}

Функция устанавливает контролу признак ошибки


\begin{longtable}[]{@{}ll@{}}
\toprule
Параметр & Описание\tabularnewline
\midrule
\endhead
bool &\tabularnewline
description & Текст ошибки\tabularnewline
id &\tabularnewline
\bottomrule
\end{longtable}

\hypertarget{setselectedvalueval-id}{%
\section{setSelectedValue(val, id)}\label{setselectedvalueval-id}}

Функция устанавливает выбранное значение для контролов типа Select

Возвращает: id элемента

\begin{longtable}[]{@{}lll@{}}
\toprule
Параметр & Тип & Описание\tabularnewline
\midrule
\endhead
val & String \textbar{} DLMap & Значение option-элемнта - элемента
списка\tabularnewline
id & & id элемента\tabularnewline
\bottomrule
\end{longtable}

\textbf{Пример}

\begin{verbatim}
setselectedvalue([a1],[f2])

Пример формы:

<form>
<select id="f2" setGroupId="group1" label="Вопрос" >
  <option id="id0empty" label=""/>
  <option id="id00" 
label="Отсутствует оплата за работу в ночное время">a1
  </option>
  <option id="id01" 
label="Отсутствует оплата за замещение должностей">a2
  </option>
</select>
</form>

Результат:

<form>
<select id="f2" setGroupId="group1" label="Вопрос" >a1
  <option id="id0empty" label=""/>
  <option id="id00" 
label="Отсутствует оплата за работу в ночное время">a1
  </option>
  <option id="id01" 
label="Отсутствует оплата за замещение должностей">a2
  </option>
</select>
</form>
\end{verbatim}

\hypertarget{setwidthval-id}{%
\section{setWidth(val, id)}\label{setwidthval-id}}

Функция устанавливает ширину контрола

Возвращает: id элемента

\begin{longtable}[]{@{}ll@{}}
\toprule
Параметр & Описание\tabularnewline
\midrule
\endhead
val & Значение ширины контрлоа\tabularnewline
id & id элемента\tabularnewline
\bottomrule
\end{longtable}

\hypertarget{gettextmap-id-dlmap}{%
\section{getText(map, id)}\label{gettextmap-id-dlmap}}

\begin{enumerate}
\def\labelenumi{\arabic{enumi}.}
\tightlist
\item
  Если элемент не select, то возвращает map где ключ и значени value
  элемента
\item
  Если элемент select и передан один аргумент - id элемента, то функция
  вернет map с text и label выбранного option- элемента
\item
  Если элемент select который не имеент выбранных option-элементов, то
  вернется map c одим пустым значением
\item
  Если первый аргумент функции map со значениями text - option элемнтов,
  то функция вернет map с text и label option- элементов
\end{enumerate}


\begin{longtable}[]{@{}l@{}}
\toprule
Param\tabularnewline
\midrule
\endhead
map\tabularnewline
id\tabularnewline
\bottomrule
\end{longtable}

\hypertarget{ismandatoryid-boolean}{%
\section{isMandatory(id)}\label{ismandatoryid-boolean}}

Возвращает обязательно ли поле для заполнения


\begin{longtable}[]{@{}ll@{}}
\toprule
Параметр & Описание\tabularnewline
\midrule
\endhead
id & id элемента\tabularnewline
\bottomrule
\end{longtable}

\hypertarget{isvisibleid-boolean}{%
\section{isVisible(id)}\label{isvisibleid-boolean}}

Возвращает видимо ли поле на форме


\begin{longtable}[]{@{}ll@{}}
\toprule
Параметр & Описание\tabularnewline
\midrule
\endhead
id & id элемента\tabularnewline
\bottomrule
\end{longtable}

\hypertarget{isenabledid-boolean}{%
\section{isEnabled(id)}\label{isenabledid-boolean}}

Возвращает доступен ли элемент для редактирования


\begin{longtable}[]{@{}ll@{}}
\toprule
Параметр & Описание\tabularnewline
\midrule
\endhead
id & id элемента\tabularnewline
\bottomrule
\end{longtable}

\hypertarget{isvalidid-boolean}{%
\section{isValid(id)}\label{isvalidid-boolean}}

Возвращает правильно ли заполенен элемент формы


\begin{longtable}[]{@{}ll@{}}
\toprule
Параметр & Описание\tabularnewline
\midrule
\endhead
id & id элемента\tabularnewline
\bottomrule
\end{longtable}

\hypertarget{isfilemandatoryid-boolean}{%
\section{isFileMandatory(id)}\label{isfilemandatoryid-boolean}}

Возвращает обязательность вложений при создании обращений

\textbf{Tod}: why func has id of element in params

\begin{longtable}[]{@{}ll@{}}
\toprule
Параметр & Описание\tabularnewline
\midrule
\endhead
id & id элемента\tabularnewline
\bottomrule
\end{longtable}

\hypertarget{isfileenabledid-boolean}{%
\section{isFileEnabled(id)}\label{isfileenabledid-boolean}}

Возвращает признак доступности вложений


\begin{longtable}[]{@{}ll@{}}
\toprule
Параметр & Описание\tabularnewline
\midrule
\endhead
id & id элемента\tabularnewline
\bottomrule
\end{longtable}

\hypertarget{clearid-string}{%
\section{clear(id)}\label{clearid-string}}

Очищает значение элемента

Возвращает: \protect\hyperlink{string}{string} - id элемента

\begin{longtable}[]{@{}lll@{}}
\toprule
Параметр & Тип & Описание\tabularnewline
\midrule
\endhead
id & String \textbar{} DLMap & id элемента\tabularnewline
\bottomrule
\end{longtable}

\hypertarget{setlist}{%
\section{setList(\_labels, \_texts, id)}\label{setlist}}

Устанавливает
option-элементы

Возвращает: \protect\hyperlink{string}{string} - id элемента

\begin{longtable}[]{@{}lll@{}}
\toprule
Параметр & Тип & Описание\tabularnewline
\midrule
\endhead
\_labels & DLMap \textbar{} \protect\hyperlink{string}{string} & строка
label разделенная pipe-символом\tabularnewline
\_texts & DLMap \textbar{} \protect\hyperlink{string}{string} & строка
text разделенная pipe-символом\tabularnewline
id & & id элемента\tabularnewline
\bottomrule
\end{longtable}

\textbf{Пример}

\begin{verbatim}
setList([label1|label2|label3],[id1|id2|id3],[this])
\end{verbatim}

\hypertarget{setmaskvalue-id}{%
\section{setMask(value, id)}\label{setmaskvalue-id}}

Устанавливает маску ввода


\begin{longtable}[]{@{}ll@{}}
\toprule
Параметр & Описание\tabularnewline
\midrule
\endhead
value & Маска ввода\tabularnewline
id & id элемента\tabularnewline
\bottomrule
\end{longtable}

\hypertarget{setfocusval-id}{%
\section{setFocus(val, id)}\label{setfocusval-id}}

Устанавливает фокус на элементе


\begin{longtable}[]{@{}lll@{}}
\toprule
Параметр & Тип & Описание\tabularnewline
\midrule
\endhead
val & boolean & Значение\tabularnewline
id & & id элемента\tabularnewline
\bottomrule
\end{longtable}

\hypertarget{setmaxvalue-id}{%
\section{setMax(value, id)}\label{setmaxvalue-id}}

Устанавливает максимальное значение для элемента


\begin{longtable}[]{@{}ll@{}}
\toprule
Параметр & Описание\tabularnewline
\midrule
\endhead
value & Максимальное значение\tabularnewline
id & id элемента\tabularnewline
\bottomrule
\end{longtable}

\hypertarget{setminvalue-id}{%
\section{setMin(value, id)}\label{setminvalue-id}}

Устанавливает минимальное значение для элемента


\begin{longtable}[]{@{}ll@{}}
\toprule
Параметр & Описание\tabularnewline
\midrule
\endhead
value & Минимальное значение\tabularnewline
id & id элемента\tabularnewline
\bottomrule
\end{longtable}

\hypertarget{stringvalue-string}{%
\section{string(value)}\label{stringvalue-string}}

Возвращает строковое значение первого параметра

Возвращает: \protect\hyperlink{string}{string} - Строковое
значение первого параметра

\begin{longtable}[]{@{}l@{}}
\toprule
Param\tabularnewline
\midrule
\endhead
value\tabularnewline
\bottomrule
\end{longtable}

\hypertarget{setbeforesavevalue-id-string}{%
\section{setBeforeSave(value, id)}\label{setbeforesavevalue-id-string}}

Устанавливает комманду, которая будет выполнена перед сохранением

Возвращает: \protect\hyperlink{string}{string} - id элемента

\begin{longtable}[]{@{}ll@{}}
\toprule
Параметр & Описание\tabularnewline
\midrule
\endhead
value & Комманда\tabularnewline
id & id элемента\tabularnewline
\bottomrule
\end{longtable}

\hypertarget{getapprovalstageid-string}{%
\section{getApprovalStage(id)}\label{getapprovalstageid-string}}

Возвращает название этапа согласования по обращению

Возвращает: \protect\hyperlink{string}{string} - Название этапа
согласования\\
\textbf{Throws}:

\begin{itemize}
\tightlist
\item
  Error(``approvalStage у внешнего обьекта не определен'')
\end{itemize}

\begin{longtable}[]{@{}ll@{}}
\toprule
Параметр & Описание\tabularnewline
\midrule
\endhead
id & id элемента\tabularnewline
\bottomrule
\end{longtable}

\hypertarget{setchecksignbool-id}{%
\section{setCheckSign(bool, id)}\label{setchecksignbool-id}}

Устанавливает признак подписания формы при сохранения обращения

\textbf{Throws}:

\begin{itemize}
\tightlist
\item
  FunctionNotImplementedError
\end{itemize}

\begin{longtable}[]{@{}ll@{}}
\toprule
Параметр & Описание\tabularnewline
\midrule
\endhead
bool &\tabularnewline
id & id элемента\tabularnewline
\bottomrule
\end{longtable}

\hypertarget{setfilemandatorybool-id-string}{%
\section{setFileMandatory(bool, id)}\label{setfilemandatorybool-id-string}}

Устаналивает признак обязательности вложения файла

Возвращает: \protect\hyperlink{string}{string} - id элемента

\begin{longtable}[]{@{}ll@{}}
\toprule
Параметр & Описание\tabularnewline
\midrule
\endhead
bool &\tabularnewline
id & id элемента\tabularnewline
\bottomrule
\end{longtable}

\hypertarget{openalerttype-message}{%
\section{openAlert(type, message)}\label{openalerttype-message}}

Открывает окно с сообщением

\textbf{Throws}:

\begin{itemize}
\tightlist
\item
  FunctionNotImplementedError
\end{itemize}

\begin{longtable}[]{@{}ll@{}}
\toprule
Параметр & Описание\tabularnewline
\midrule
\endhead
type & Тип сообщения info, warning, error\tabularnewline
message & Текст сообщения\tabularnewline
\bottomrule
\end{longtable}

\hypertarget{setsaveenabledbool-id-string}{%
\section{setSaveEnabled(bool, id)}\label{setsaveenabledbool-id-string}}

Устанавливает признак доступности кнопкаи ``Сохранить''

Возвращает: \protect\hyperlink{string}{string} - id элемента

\begin{longtable}[]{@{}ll@{}}
\toprule
Параметр & Описание\tabularnewline
\midrule
\endhead
bool &\tabularnewline
id & id элемента\tabularnewline
\bottomrule
\end{longtable}

\hypertarget{setfileenabledbool-id-string}{%
\section{setFileEnabled(bool, id)}\label{setfileenabledbool-id-string}}

Устанавливает признак доступности кнопки ``Вложить''

Возвращает: \protect\hyperlink{string}{string} - id элемента

\begin{longtable}[]{@{}ll@{}}
\toprule
Параметр & Описание\tabularnewline
\midrule
\endhead
bool &\tabularnewline
id & id элемента\tabularnewline
\bottomrule
\end{longtable}

\hypertarget{setexternalsystemextsystem-id-string}{%
\section{setExternalSystem(extsystem, id)}\label{setexternalsystemextsystem-id-string}}

Устанавливает код внешней системы

Возвращает: \protect\hyperlink{string}{string} - id элемента

\begin{longtable}[]{@{}lll@{}}
\toprule
\begin{minipage}[b]{0.30\columnwidth}\raggedright
Param\strut
\end{minipage} & \begin{minipage}[b]{0.30\columnwidth}\raggedright
Type\strut
\end{minipage} & \begin{minipage}[b]{0.30\columnwidth}\raggedright
Description\strut
\end{minipage}\tabularnewline
\midrule
\endhead
\begin{minipage}[t]{0.30\columnwidth}\raggedright
extsystem\strut
\end{minipage} & \begin{minipage}[t]{0.30\columnwidth}\raggedright
SM \textbar{} FRIEND\strut
\end{minipage} & \begin{minipage}[t]{0.30\columnwidth}\raggedright
Код внешней системы SM или FRIEND\strut
\end{minipage}\tabularnewline
\begin{minipage}[t]{0.30\columnwidth}\raggedright
id\strut
\end{minipage} & \begin{minipage}[t]{0.30\columnwidth}\raggedright
\strut
\end{minipage} & \begin{minipage}[t]{0.30\columnwidth}\raggedright
id элемента\strut
\end{minipage}\tabularnewline
\bottomrule
\end{longtable}

\hypertarget{setinformationvisiblebool-id-string}{%
\section{setInformationVisible(bool, id)}\label{setinformationvisiblebool-id-string}}

Устанавливает признак видимости поля ``Информация''

Возвращает: \protect\hyperlink{string}{string} - id элемента

\begin{longtable}[]{@{}ll@{}}
\toprule
Параметр & Описание\tabularnewline
\midrule
\endhead
bool &\tabularnewline
id & элемента\tabularnewline
\bottomrule
\end{longtable}

\hypertarget{setinitiatorvalue-id-string}{%
\section{setInitiator(value, id)}\label{setinitiatorvalue-id-string}}

Устанавливает инициатора в обращении

Возвращает: \protect\hyperlink{string}{string} - id элемента

\begin{longtable}[]{@{}ll@{}}
\toprule
Параметр & Описание\tabularnewline
\midrule
\endhead
value & Идентификатор инициатора\tabularnewline
id & id элемента\tabularnewline
\bottomrule
\end{longtable}

\hypertarget{setmultiflowbool-id}{%
\section{setMultiFlow(bool, id)}\label{setmultiflowbool-id}}

Устанавливает признак мультипоточного ввода по шаблону

Возвращает: * - id элемента

\begin{longtable}[]{@{}ll@{}}
\toprule
Параметр & Описание\tabularnewline
\midrule
\endhead
bool &\tabularnewline
id & id элемента\tabularnewline
\bottomrule
\end{longtable}

\hypertarget{setredirectvalue-id-skipconvert}{%
\section{setRedirect(value, id, skipConvert)}\label{setredirectvalue-id-skipconvert}}

Устанавливает ссылку на которую надо перейти

Возвращает: * - id элемента

\begin{longtable}[]{@{}lll@{}}
\toprule
Параметр & Тип & Описание\tabularnewline
\midrule
\endhead
value & \protect\hyperlink{string}{string} & Ссылка\tabularnewline
id & \protect\hyperlink{string}{string} & id элемента\tabularnewline
skipConvert & Boolean & не преобразовывать ссылку LD1 \textgreater{}
LD2\tabularnewline
\bottomrule
\end{longtable}

\hypertarget{setrelationshipobjectid-objecttype-id}{%
\section{setRelationship(objectId, objectType, id)}\label{setrelationshipobjectid-objecttype-id}}

Используется только с ПОДРУГа! Функция устанавливает связь созданного
обращения с обьектом типа objectType c номером objectId

Возвращает: id элемента

\begin{longtable}[]{@{}ll@{}}
\toprule
Параметр & Описание\tabularnewline
\midrule
\endhead
objectId & Номер обьекта\tabularnewline
objectType & Тип обьекта 
ZNO/ZNU\tabularnewline
id & id элемента\tabularnewline
\bottomrule
\end{longtable}

\hypertarget{setadditionaltemplateextsystem-templateid-id}{%
\section{setAdditionalTemplate}\label{setadditionaltemplateextsystem-templateid-id}}

Устанавливает шаблона по которому нужно дополнительно создать обращение

\textbf{Throws}:

\begin{itemize}
\tightlist
\item
  FunctionNotImplementedError
\end{itemize}

\begin{longtable}[]{@{}lll@{}}
\toprule
\begin{minipage}[b]{0.30\columnwidth}\raggedright
Param\strut
\end{minipage} & \begin{minipage}[b]{0.30\columnwidth}\raggedright
Type\strut
\end{minipage} & \begin{minipage}[b]{0.30\columnwidth}\raggedright
Description\strut
\end{minipage}\tabularnewline
\midrule
\endhead
\begin{minipage}[t]{0.30\columnwidth}\raggedright
extsystem\strut
\end{minipage} & \begin{minipage}[t]{0.30\columnwidth}\raggedright
SM \textbar{} FRIEND\strut
\end{minipage} & \begin{minipage}[t]{0.30\columnwidth}\raggedright
Код внешней системы SM или FRIEND\strut
\end{minipage}\tabularnewline
\begin{minipage}[t]{0.30\columnwidth}\raggedright
templateId\strut
\end{minipage} & \begin{minipage}[t]{0.30\columnwidth}\raggedright
\strut
\end{minipage} & \begin{minipage}[t]{0.30\columnwidth}\raggedright
Идентификатор шаблона обращения\strut
\end{minipage}\tabularnewline
\begin{minipage}[t]{0.30\columnwidth}\raggedright
id\strut
\end{minipage} & \begin{minipage}[t]{0.30\columnwidth}\raggedright
\strut
\end{minipage} & \begin{minipage}[t]{0.30\columnwidth}\raggedright
id элемента\strut
\end{minipage}\tabularnewline
\bottomrule
\end{longtable}

\hypertarget{setscoringvisiblebool-id-string}{%
\section{setScoringVisible(bool, id)}\label{setscoringvisiblebool-id-string}}

Устанавливает признак видимости поля <<Скоринг>> в обращении


\begin{longtable}[]{@{}ll@{}}
\toprule
Параметр & Описание\tabularnewline
\midrule
\endhead
bool &\tabularnewline
id & id элемента\tabularnewline
\bottomrule
\end{longtable}

\hypertarget{getsysdatevalue}{%
\section{getSysDate(value)}\label{getsysdatevalue}}

Возвращает смещение относительно текущей даты

Возвращает: Дата

\begin{longtable}[]{@{}ll@{}}
\toprule
Параметр & Описание\tabularnewline
\midrule
\endhead
value & Смещение в днях\tabularnewline
\bottomrule
\end{longtable}

\textbf{Пример}

\begin{verbatim}
getSysDate([+5]) - вернет текущую дату + 5 дней
\end{verbatim}

\hypertarget{getinitiatorvalue-string}{%
\section{getInitiator(value)}\label{getinitiatorvalue-string}}

Получение информации об инициаторе обращения

Возвращает: \protect\hyperlink{string}{string} - Информация по
параметру

\begin{longtable}[]{@{}ll@{}}
\toprule
\begin{minipage}[b]{0.47\columnwidth}\raggedright
Param\strut
\end{minipage} & \begin{minipage}[b]{0.47\columnwidth}\raggedright
Description\strut
\end{minipage}\tabularnewline
\midrule
\endhead
\begin{minipage}[t]{0.47\columnwidth}\raggedright
value\strut
\end{minipage} & \begin{minipage}[t]{0.47\columnwidth}\raggedright
Параметр fio, domaincode, ou, tabnum, id, departcode, position, guid,
email, vsp, tbcode\strut
\end{minipage}\tabularnewline
\bottomrule
\end{longtable}

\hypertarget{setminhoursvalue-id}{%
\section{setMinHours(value, id)}\label{setminhoursvalue-id}}

Устаналивает минимальное значение в часах

Возвращает: * - id элемента

\begin{longtable}[]{@{}ll@{}}
\toprule
Параметр & Описание\tabularnewline
\midrule
\endhead
value & Значение часов в 24 - часовом формате\tabularnewline
id & id элемента\tabularnewline
\bottomrule
\end{longtable}

\hypertarget{setmaxhoursvalue-id}{%
\section{setMaxHours(value, id)}\label{setmaxhoursvalue-id}}

Устаналивает максимальное значение в часах

Возвращает: * - id элемента

\begin{longtable}[]{@{}ll@{}}
\toprule
Параметр & Описание\tabularnewline
\midrule
\endhead
value & Значение часов в 24 - часовом формате\tabularnewline
id & id элемента\tabularnewline
\bottomrule
\end{longtable}

\hypertarget{setcontaincustomobjectsbool-id}{%
\section{setContainCustomObjects(bool, id)}\label{setcontaincustomobjectsbool-id}}

Устанавливает признак того что созданное обращение будет содержать
``хранимые'' обьекты, например обьет закупки, поездки и т.д.


\begin{longtable}[]{@{}l@{}}
\toprule
Param\tabularnewline
\midrule
\endhead
bool\tabularnewline
id\tabularnewline
\bottomrule
\end{longtable}

\hypertarget{getsizeparam}{%
\section{getSize(param)}\label{getsizeparam}}

Возвращает количество option-элементов в элементе если передан
идентификатор элемента или количество полей в DLMap если передана DLMap

Возвращает: * - id элемента

\begin{longtable}[]{@{}lll@{}}
\toprule
Параметр & Тип & Описание\tabularnewline
\midrule
\endhead
param & \protect\hyperlink{string}{string} \textbar{} DLMap & id
элемента или DLMap\tabularnewline
\bottomrule
\end{longtable}

\hypertarget{setstepvalue-id}{%
\section{setStep(value, id)}\label{setstepvalue-id}}

Устанавливает шаг для выбора значения

Возвращает: * - id элемента

\begin{longtable}[]{@{}ll@{}}
\toprule
Параметр & Описание\tabularnewline
\midrule
\endhead
value & Размер шага\tabularnewline
id & id элемента\tabularnewline
\bottomrule
\end{longtable}

\hypertarget{setstylearg-value-id-string}{%
\section{setStyle(arg, value, id)}\label{setstylearg-value-id-string}}

Устанавливает стиль для атрибута

Поддерживаемые стили:

color - цвет элемента badge - текст badge - элемента badge-color - цвет
badge - элемента width - ширина элемнта (только web-версии Лица ДРУГА)

Возвращает: \protect\hyperlink{string}{string} - id элемента

\begin{longtable}[]{@{}ll@{}}
\toprule
Параметр & Описание\tabularnewline
\midrule
\endhead
arg & Стиль\tabularnewline
value & Значение\tabularnewline
id & id элемента\tabularnewline
\bottomrule
\end{longtable}

\hypertarget{compareleft-operand-rigth}{%
\section{compare(left, operand,
rigth)}\label{compareleft-operand-rigth}}

Функция сравнения

Доступны операторы сравнения \textgreater, \textgreater=, ==, \textless,
\textless=


\begin{longtable}[]{@{}ll@{}}
\toprule
Параметр & Описание\tabularnewline
\midrule
\endhead
left & Перый аргумент для сравнения\tabularnewline
operand & Оператор сравнения\tabularnewline
rigth & Второй аргумент для сравнения\tabularnewline
\bottomrule
\end{longtable}

\hypertarget{setinitiatorvisiblebool-id-string}{%
\section{setInitiatorVisible(bool, id)}\label{setinitiatorvisiblebool-id-string}}

Устаналивает признак видимости поля смены ВК

Возвращает: \protect\hyperlink{string}{string} - id элемента

\begin{longtable}[]{@{}ll@{}}
\toprule
Параметр & Описание\tabularnewline
\midrule
\endhead
bool &\tabularnewline
id & id элемента\tabularnewline
\bottomrule
\end{longtable}

\hypertarget{setdescriptiontext-id-string}{%
\section{setDescription(text, id)}\label{setdescriptiontext-id-string}}

Устанавливает лейбл у элемента

Возвращает: String - Идентификатор элемента

\begin{longtable}[]{@{}lll@{}}
\toprule
Параметр & Тип & Описание\tabularnewline
\midrule
\endhead
text & String & Тексе лейбла\tabularnewline
id & String & Идентификатор элемента\tabularnewline
\bottomrule
\end{longtable}

\hypertarget{getrelatedid-string}{%
\section{getRelatedId()}\label{getrelatedid-string}}

Возвращает идентификатор связанного обращения

Возвращает: String - Идентификатор связанного обращения\\

\hypertarget{setcopyvisiblebool-id-string}{%
\section{setCopyVisible(bool, id)}\label{setcopyvisiblebool-id-string}}

Устанавливает видимость кнопки копирования шаблона

Возвращает: String - Возвращает второй аргумент

\begin{longtable}[]{@{}ll@{}}
\toprule
Параметр & Тип\tabularnewline
\midrule
\endhead
bool & Boolean\tabularnewline
id & String\tabularnewline
\bottomrule
\end{longtable}

\hypertarget{setdeclinevisiblebool-id-string}{%
\section{setDeclineVisible(bool, id)}\label{setdeclinevisiblebool-id-string}}

Устанавливает видимость кнопки ``Отменить обращение''

Возвращает: String - Возвращает второй аргумент

\begin{longtable}[]{@{}ll@{}}
\toprule
Параметр & Тип\tabularnewline
\midrule
\endhead
bool & Boolean\tabularnewline
id & String\tabularnewline
\bottomrule
\end{longtable}

\hypertarget{setclosevisiblebool-id-string}{%
\section{setCloseVisible(bool, id)}\label{setclosevisiblebool-id-string}}

Устанавливает видимость кнопки ``Закрыть группу''

Возвращает: String - Возвращает второй аргумент

\begin{longtable}[]{@{}ll@{}}
\toprule
Параметр & Тип\tabularnewline
\midrule
\endhead
bool & Boolean\tabularnewline
id & String\tabularnewline
\bottomrule
\end{longtable}

\hypertarget{getclienttype-string}{%
\section{getClientType()}\label{getclienttype-string}}

Возвращает тип клиента. Возможные значение: ios, android, web-sigma,
web-alpha

Возвращает: \protect\hyperlink{string}{string} - Тип клиента\\

\hypertarget{ismobileclient-boolean}{%
\section{isMobileClient()}\label{ismobileclient-boolean}}

Возвращает является ли клиент мобильным приложением


\hypertarget{iswebclient-boolean}{%
\section{isWebClient()}\label{iswebclient-boolean}}

Возвращает является ли клиент web-приложением


\hypertarget{isiosclient-boolean}{%
\section{isIOSClient()}\label{isiosclient-boolean}}

Возвращает является ли клиент iOs-приложением


\hypertarget{isandroidclient-boolean}{%
\section{isAndroidClient()}\label{isandroidclient-boolean}}

Возвращает является ли клиент Android-приложением


\hypertarget{iswebsigmaclient-boolean}{%
\section{isWebSigmaClient()}\label{iswebsigmaclient-boolean}}

Возвращает является ли клиент web sigma приложением


\hypertarget{iswebalphaclient-boolean}{%
\section{isWebAlphaClient()}\label{iswebalphaclient-boolean}}

Возвращает является ли клиент web alpha приложением


\hypertarget{isface20}{%
\section{isFace20()}\label{isface20}}

Функция возвращает true если это web прложение face20 Костыль нужен для
авто редиректа на /friendface если шаблон не работает на face20.
Оставляем пока face20 не стабилизируется.


\hypertarget{setactionvalue-id}{%
\section{setAction(value, id)}\label{setactionvalue-id}}

Устанавливает значение в атрибут sbaction элемента

Возвращает: id

\begin{longtable}[]{@{}l@{}}
\toprule
Param\tabularnewline
\midrule
\endhead
value\tabularnewline
id\tabularnewline
\bottomrule
\end{longtable}

\hypertarget{setmodevalue-id}{%
\section{setMode(value, id)}\label{setmodevalue-id}}

Устанавливает значение в атрибут sbmode элемента

Возвращает: id

\begin{longtable}[]{@{}l@{}}
\toprule
Param\tabularnewline
\midrule
\endhead
value\tabularnewline
id\tabularnewline
\bottomrule
\end{longtable}

\hypertarget{datediffstart-end-id}{%
\section{dateDiff(start, end, id)}\label{datediffstart-end-id}}

Возвращает разницу в днях


\begin{longtable}[]{@{}l@{}}
\toprule
Param\tabularnewline
\midrule
\endhead
start\tabularnewline
end\tabularnewline
id\tabularnewline
\bottomrule
\end{longtable}

\hypertarget{iscreatemode-boolean}{%
\section{isCreateMode()}\label{iscreatemode-boolean}}

Форма в режиме создания обращения


\hypertarget{isopenmode-boolean}{%
\section{isOpenMode()}\label{isopenmode-boolean}}

Форма в режиме созданного обращения


\hypertarget{setsavevisiblebool-id-string}{%
\section{setSaveVisible(bool, id)}\label{setsavevisiblebool-id-string}}

Устанавливает видимость кнопки сохранения шаблона

Возвращает: String - Возвращает второй аргумент

\begin{longtable}[]{@{}ll@{}}
\toprule
Параметр & Тип\tabularnewline
\midrule
\endhead
bool & Boolean\tabularnewline
id & String\tabularnewline
\bottomrule
\end{longtable}
\emergencystretch 3em
\hypertarget{filename-templatename-extension}{%
\section{createDocument}\label{filename-templatename-extension}}

Заполняет шаблон документа (templateName) данными из обращения и
возвращает его в качестве вложения с названием fileName.extension
\textbar{} Param \textbar{} Type \textbar{} Description \textbar{}
\textbar{} --- \textbar{} --- \textbar{} --- \textbar{} \textbar{}
fileName \textbar{} String \textbar{} Имя возвращаемого вложенияя
\textbar{} \textbar{} templateName \textbar{} String \textbar{} Имя
шаблона документа, куда будут подставлены данные из обращения \textbar{}
\textbar{} extension \textbar{} String \textbar{} Расширение
возвращаемого вложения (доступные значения PDF). Не обязательный
параметр \textbar{}

Пример:

\begin{verbatim}
setValue(
  [createDocument(
    [Заявление], [ЗаявлениеШаблон], [PDF])], 
  [attachment_el_id]
)

\end{verbatim}

Делегирует обращение к серверу и формирование документа-вложения функции
doCreateDocument(tid, params, callback) Пример:

\begin{verbatim}
DOMCore.doCreateDocument = (tid, params, callback) => {
  //params.fileName - имя возвращаемого вложения
  //params.templateName - имя шаблона документа
  //params.extension - расширение возвращаемого вложения
  //params.interaction - объект Interaction
  const response = anyDelegateFunction(params.fileName, 
  params.templateName, params.extension, params.interaction);
  callback(tid, response, null);
}
\end{verbatim}

Функция доступна из JS шаблона через глобальный вызов doCreateDocument
Пример:

\begin{verbatim}
var fileName = 'Заявление';
var templateName = 'ЗаявлениеШаблон';
var extension = 'PDF';
doCreateDocument(fileName, templateName, 
  extension, onSuccess, onError);

var onSuccess = function(attachmentData) {
  console.log(JSON.stringify(attachmentData));
}

var onError = function(err) {
  console.error(err && err.userMessage);
}
\end{verbatim}

\hypertarget{attachment-mode}{%
\section{signAttachment(attachment,mode)}\label{attachment-mode}}

Запускает процесс подписи вложения (attachment) и возвращает подписанное
вложение


\begin{longtable}[]{@{}lll@{}}
\toprule
\begin{minipage}[b]{0.30\columnwidth}\raggedright
Param\strut
\end{minipage} & \begin{minipage}[b]{0.30\columnwidth}\raggedright
Type\strut
\end{minipage} & \begin{minipage}[b]{0.30\columnwidth}\raggedright
Description\strut
\end{minipage}\tabularnewline
\midrule
\endhead
\begin{minipage}[t]{0.30\columnwidth}\raggedright
attachment\strut
\end{minipage} & \begin{minipage}[t]{0.30\columnwidth}\raggedright
String\strut
\end{minipage} & \begin{minipage}[t]{0.30\columnwidth}\raggedright
Объект вложения(й) (AttachmentDTO) сериализованный в JSON\strut
\end{minipage}\tabularnewline
\begin{minipage}[t]{0.30\columnwidth}\raggedright
mode\strut
\end{minipage} & \begin{minipage}[t]{0.30\columnwidth}\raggedright
String\strut
\end{minipage} & \begin{minipage}[t]{0.30\columnwidth}\raggedright
Режим подписи (доступные значения CLOUD - облачная подпись, TABLET -
подпись через TM)\strut
\end{minipage}\tabularnewline
\bottomrule
\end{longtable}

Пример:

\begin{verbatim}
setValue(
  [signAttachment(
    [getValue([nonsignedattachment])],
    [CLOUD]
  )],
  [attachmentelement]
)
\end{verbatim}

Делегирует подписание функции 

doSignAttachment(tid,
signAttachmentParams, callback) 

Пример:

\begin{verbatim}
DOMCore.doSignAttachment = (tid, params, callback) => {
  // params.attachment - подписываемое вложения
  // params.signMethod - метод подписания (CLOUD, TABLET)
  const response = anyDelegateFunction(params);
  callback(tid, response, null);
}
\end{verbatim}

Функция доступна из JS шаблона через глобальный вызов

\begin{verbatim}
doSignAttachment(attachment, method, onSuccess, onError) 
\end{verbatim}

Пример:
\begin{verbatim}
var attachment = {asyncId: 12345};
var method = 'CLOUD';
doSignAttachment(attachment, method, onSuccess, onError);

var onSuccess = function(attachmentData) {
  console.log(JSON.stringify(attachmentData));
}

var onError = function(err) {
  console.error(err && err.userMessage);
}
\end{verbatim}

