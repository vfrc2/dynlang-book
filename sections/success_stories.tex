\documentclass[../index.tex]{subfiles}

\begin{document}

	\section{Интервью с Александром Носовым}
	
	Т:  Привет, расскажи, помнишь ли ты день встречи с Кириллом? Как это было?
	
	А:  Конечно! Это было лето 2012. Мы тогда устроились на работу с Кириллом примерно в один месяц. Кирилл занимался Бенк Хелпером. Сразу было видно, что человек целеустремленный, настойчивый. Он свернул гору непонимания и внедрил свой продукт в центральном аппарате. Кирилл был нереально крут: он был  и админом и внедренцем и сопровожденцем всей этой истории и надо сказать, Бенк Хелпер у него просто взлетел! 
	
	
	Т:  Помнишь ли ты, как динамический язык вошел в твою жизнь?
	
	А:  О, это был не день.. это было пару месяцев. В то время я занимался созданием серверной части портала самообслуживания, а Кирилл как раз занимался Web-частью, где как раз самостоятельно внедрил свой язык. В мою жизнь он так и не вошел, но радость у людей от его использования была очевидна, в глазах был кайф, что можно сделать классные шаблончики для пользователей. 
	
	
	
	Т: И всё таки, как думаешь отразился ли на твоей жизни динамический язык?
	
	А:  Да, пожалуй, пониманием того, что всё что кажется сложным и невозможным — не всегда таким является. Тот же Авито рассказывает, что придумал свой супер-крутой динамический язык, но, на мой взгляд, он даже рядом не стоял с тем, что сделал Кирилл. Потому что никто не мог себе позволить даже  подумать, что это возможно прямо на клиенте устраивать подобные махинации. 
	
	Т:  Какое развитие ты видишь у Динамического языка?
	
	А:  Динамический язык развивается уже самостоятельно, в нем появляются радикалы и участки кода, который сами с собой взаимодействуют, и даже Кирилл не всегда понимает как они работают.
	
	Развитие языка я вижу в его перерождении в нечто более современное, возможно dom-core, что упростит вхождение для новых разработчиков и позволит использовать преимущества языка, например JS, и тогда динамический язык станет еще более динамическим!
	
	\section{Юлия Кувинова и её детская мечта}
	
	В 2016 году моя детская мечта реализовалась, я встретила Дон Кихота Ломанческого героя романа Мигеля де Сервантеса по имени Кирилл. Самые лучшие идеи спасения мира, защиты слабых, заботы о тех, кто не умеет дружить с современной техникой и программами приходят в эту восхитительную голову в минуты стресса и одновременного вдохновения. Бесстрашие и отчаянная борьба за лучший мир движет эту великолепную натуру на борьбу с мельничными жерновами процессов, отношений и несогласий. Энтузиазм и бесконечная изобретательность, в сочетании с всеобъемлющей добротой и преданностью делу, друзьям и семье восторгает окружающих людей. Кирилл - прекрасен, сталкиваясь с не решаемыми задачами, в том числе этическими, понимаю, как Кирилл спасает меня, каждый день своим примером, бескорыстием и энергией, которую он раздаёт всем, как солнце, нагревающее эту планету. Как Дульсинея Тобосская/ Альдонса Лоренсо/ Юлька Кувинова восторгаюсь и скучаю о странствующем рыцаре средневекового романа моего детства.
	
	
	\section{Вячеслав Гавришин в стране оптимизма}
	
	Я встретил Кирилла в 2014 году и с тех пор оптимизма и Ростовского размаха в моей жизни прибавилось. Череда приятных событий изменила меня в лучшую сторону -- я осознал себя в новом лице, а потом у меня появилась ещё одна подруга. Правда с правдой не сразу заладилось - пришлось переделывать. Но зато форсаж получился отменный - мы стали гибче. Оглядываясь на все это, я с уверенностью смотрю в будущее и оно прекрасно -- у Дурга будет новый мозг и новая среда. Все получится!
	
	
\end{document}