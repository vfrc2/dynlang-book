\documentclass[../index.tex]{subfiles}


\newglossaryentry{command}
{
    name=Команда,
    description={Одна или более функций динамического языка объединенных в последовательный блок выполнения(в рамках процедурного подхода команду можно интерпретировать как процедуру)}
}

\begin{document}
\section{Что такое программа}
    Программа - термин, в переводе означающий «предписание», то есть заданную последовательность действий. В динамическом языке программой принято называть последовательность комманд(\gls{command}) записанную в формате функций с аргументами.
    Каждый аргумент указывается в квадратных скобочках. Аргументы между собой разделяется запятыми (например: скажи([привет],[как],[дела]) ). Основная задача программ в динамическом языке - управление состоянием и значениями элементов динамической формы(подробнее в разделе \hyperref[sec:dynfom]{Динамическая форма}).
    Должная сноровка и немного усилий после изучения данной книги позволит написать вам программу решающую любой бизнес-кейс возникающий на просторах вселенной. До сегодняшнего дня никто не знает пределов возможностей динамического языка.

    \subsection{Основные концепции}
        Предугадав тенденции развития функциональных языков программирования, динамический язык взял за основу концепцию именно данного типа языков, по факту в динамическом языке отсутствуют переменные, константы и остальные нагромождения ООП, все реализуется через функции или через чистые функции. Хотя результатом функций и являются объекты с которыми происходит взаимодействие, но по мнению автора(ов) при разработке программ на динамическом языке стоит придерживаться функциональных подходов программирования. Это позволит полноценно оценить все тонкости разработки.
        Константы и строки в динамическом языке являются чистыми функциями в результате возвращающими литеральное значение самих себя.
                
        В частности чистая функция \texttt{this} или 
        \texttt{\_\_this\_\_} возвращает объект над которым производится вычисление функции
        А функция \texttt{text} вернет строковый литерал соответствующий самому себе.
        
        Вышесказанные концепции очень важны для дальнейшего понимания функционального ядра динамического языка.

	\section{Запуск динамических команд}
	Одним из первых вызовов, который бросает динамический язык начинающему разработчику, является необходимость настройки рабочего окружения HP Service Manager для редактирования шаблонов динамических форм. Однако, благодаря мощным возможностям динамического языка, сделать первые шаги в разработке можно гораздо проще: достаточно открыть шаблон обращения <<Исполнятор>> в dev-среде портала <<Лицо ДРУГа>>. Это простая динамическая форма, состящая всего из двух полей: <<Команда>> и <<Результат>>.
	
	\section{Первая программа}
	
	По давней традиции, первая программа, которую начинающий разработчик пишет на новом языке, является <<Hello, world!>>. На динамическом языке она выглядит так:
	\begin{verbatim}
	>>> setValue([Hello, world!],[result])
	\end{verbatim}
	
	Для выполнения программы введите ее в поле <<Команда>> и переставьте курсор в поле <<Результат>>. В итоге, в поле <<Результат>> вы увидите текст:
	
	\begin{verbatim}
	Hello, World!
	\end{verbatim}
	
	Поздравляем! Вы сделали первый шаг на безумно интересном пути к освоению могучих возможностей создани интерактивных форм, который дает вам Динамический язык.
	Это пример команды {\bf setValue}, которая устанавливает значение, указанное первым аргументом в квадратных скобках в поле, указанное во втором аргументе.
	
	Здесь и далее, все команды, которые вводятся в поле <<Команда>> будут предваряться символами 	\begin{verbatim}
	>>>
	\end{verbatim}	За которыми следует результат выполнения в поле <<Результат>>
	
	\section{Чувствительность к регистру}
	
	Динамческий язык нечувствителен к регистру. Это означает, что программы:
	
	\begin{verbatim}
	>>> setvalue([Hello, world!],[result])
	\end{verbatim}
	
	и
	
	\begin{verbatim}
	>>> setvalue([Hello, world!],[result])
	\end{verbatim}
	
	и даже
	
	\begin{verbatim}
	>>> sEtVaLuE([Hello, world!],[result])
	\end{verbatim}
	
	C точки зрения интерпретатора динамического языка являются индентичными и произведут одинаковый результат. И несмотря на то, что последний пример выглядит, безусловно, наиболее круто, в примерах к книге мы будем придерживаться так называемого CamelCase, так как он наиболее удобочитаем:
	
	\begin{verbatim}
	>>> setValue([Hello, world!],[result])
	\end{verbatim}
	
	\section{Арифметические функции}
	
	Перейдем от <<Hello, world!>> к арифметике. Как и любой, уважающий себя язык программирования, динамический язык предоставляет возможность работы с основными арифметическими операциями, а именно {\bf plus}, {\bf minus} и {\bf mul} для сложения, вычитания и умножения, а также {\bf div} для деления:
	
	\begin{verbatim}
	>>> setValue([plus([68],[1])],[result])
	69
	>>> setValue([minus([69],[1])],[result])
	68
	>>> setValue([mul([6],[7])],[result])
	42
	>>> setValue([div([84],[2])],[result])
	42
	\end{verbatim}
	
	На самом деле, указанные функции обладают гораздо большим набором возможностей, но для первого знакомства этого будет достаточно. 
				
	\section{Упражнения}
	
	Откройте шаблон <<Исполнятор>> и решите следующие задачи:
	
	\begin{enumerate}
		\item Сколько будет в градусах Цельсия температура в 69 Фаренгейт?
		
		\item Сколько спринтов потребуется для вывода в промышленную эксплуатацию шаблона <<Совместная поездка>> если каждый спринт исправляется 6 багов, и обнаруживается 4 новых?
    \end{enumerate}
\end{document}