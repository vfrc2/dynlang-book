\documentclass[../index.tex]{subfiles}
 
\newglossaryentry{command}
{
    name=Команда,
    description={Одна или более функций динамического языка объединенных в последовательный блок выполнения(в рамках процедурного подхода команду можно интерпретировать как процедуру)}
}

\begin{document}
\section{Что такое программа}

    Программа -- комбинация компьютерных инструкций и данных, позволяющая аппаратному обеспечению вычислительной системы выполнять вычисления или функции управления. 
    
    Как правило, компьютерная программа написана программистом на языке программирования. Из программы, написанной на понятном человеку исходном коде, комьютер может сформировать машинный код -- набор инструкций, который компьютер может исполнять напрямую. В качестве альтернативы, программа может быть исполнена при помощи интерпретатора.
    
    Несмотря на различия между множеством существующих языков программирования, некоторые базовые инструкции пристуствуют почти в каждом из них:
    
    \textbf{ввод}: получить данные с клавиатуры, из файла или другого устройства
    
    \textbf{вывод}: отобразить данные на экране или отправить на другое устройство
    
    \textbf{арифметика}: произвести базовые математические операции, такие как сложение или умножение
    
    \textbf{ветвление}: проверить определенное состояние и выполнить соответствующий ему код
    
    \textbf{повторение}: произвести определенное действие неоднократно, иногда с некоторыми вариациями
        
        
    В динамическом языке программой принято называть последовательность команд, записанную в формате функций с аргументами.
    Каждый аргумент указывается в квадратных скобочках. Аргументы между собой разделяются запятыми, например: 
    \begin{verbatim}
    скажи([привет],[как],[дела])
    \end{verbatim}

    \section{Сценарные языки программирования}
        Сценарный язык (язык сценариев, жарг. скриптовый язык; англ. scripting language) -- высокоуровневый язык программирования, автоматизирующий выполнение задач в специализированной среде выполнения. 
        Как правило, сценарные языки являются интерпретируемыми, а не компилируемыми. Сценарный язык 
        можно рассматривать как предметно-ориентированный язык (англ. domain-specific language, DSL — «язык, специфический для предметной области») для определенного окружения.
        
        \emergencystretch 3em
        Язык программирования динамических форм (далее \textbf{Динамический язык}) -- сценарный предметно-ориентированный язык, предназначнный для автоматизации управления состоянием 
        интерактивных форм регистрации обращений в экосистеме ДРУГа.
        
	\section{Запуск динамических команд}
	Одним из первых вызовов, который бросают динамические формы начинающему разработчику, является необходимость настройки рабочего окружения HP Service Manager для редактирования шаблонов. Однако, благодаря мощным возможностям динамического языка, сделать первые шаги в разработке можно гораздо проще: достаточно открыть шаблон обращения <<Исполнятор>> в dev-среде портала <<Лицо ДРУГа>>. Это простая динамическая форма, состящая всего из двух полей: <<Команда>> и <<Результат>>.
	
	\section{Первая программа}
		
	\epigraph{Talk is cheap. Show me the code}{\textit{Linus Torvalds}}
	
	По давней традиции, первая программа, которую начинающий разработчик пишет на новом языке, является <<Hello, world!>>. На динамическом языке она выглядит так:
	\begin{verbatim}
	>>> setValue([Hello, world!],[result])
	\end{verbatim}
	
	Для выполнения программы введите ее в поле <<Команда>> и переставьте курсор в поле <<Результат>>. В итоге, в поле <<Результат>> вы увидите текст:
	
	\begin{verbatim}
	Hello, World!
	\end{verbatim}
	
	Поздравляем! Вы сделали первый шаг на безумно интересном пути к освоению могучих возможностей создания интерактивных форм, который дает вам Динамический язык.
	Это пример команды {\bf setValue}, которая устанавливает значение, указанное первым аргументом в квадратных скобках в поле, указанное во втором аргументе.
	
	Здесь и далее, все команды, которые вводятся в поле <<Команда>> будут предваряться символами 	\begin{verbatim}
	>>>
	\end{verbatim}	За которыми следует результат выполнения в поле <<Результат>>
	
	\section{Чувствительность к регистру}
	
	Динамческий язык нечувствителен к регистру. Это означает, что программы:
	
	\begin{verbatim}
	>>> setvalue([Hello, world!],[result])
	\end{verbatim}
	
	и
	
	\begin{verbatim}
	>>> SETVALUE([Hello, world!],[result])
	\end{verbatim}
	
	и даже
	
	\begin{verbatim}
	>>> sEtVaLuE([Hello, world!],[result])
	\end{verbatim}
	
	C точки зрения интерпретатора динамического языка являются индентичными и произведут одинаковый результат. И несмотря на то, что последний пример выглядит, безусловно, наиболее круто, в примерах к книге мы будем придерживаться так называемого CamelCase, так как он наиболее удобочитаем:
	
	\begin{verbatim}
	>>> setValue([Hello, world!],[result])
	\end{verbatim}
	
	\section{Арифметические функции}
	
	Перейдем от <<Hello, world!>> к арифметике. Как и любой, уважающий себя язык программирования, динамический язык предоставляет возможность работы с основными арифметическими операциями, а именно {\bf plus}, {\bf minus} и {\bf mul} для сложения, вычитания и умножения соотвественно, а также {\bf div} для деления:
	
	\begin{verbatim}
	>>> setValue([plus([68],[1])],[result])
	69
	>>> setValue([minus([69],[1])],[result])
	68
	>>> setValue([mul([6],[7])],[result])
	42
	>>> setValue([div([84],[2])],[result])
	42
	\end{verbatim}
	
	На самом деле, указанные функции обладают гораздо большим набором возможностей, но для первого знакомства этого будет достаточно. 
				
	\section{Упражнения}
	
	Откройте шаблон <<Исполнятор>> и, при помощи динамического языка, решите следующие задачи:
	
	\begin{enumerate}
		\item Сколько будет в градусах Цельсия температура в 69 Фаренгейт?
		
		\item Сколько спринтов потребуется для вывода в промышленную эксплуатацию шаблона <<Совместная поездка>> если каждый спринт исправляется 4 багов, и обнаруживается 6 новых?
    \end{enumerate}
\end{document}