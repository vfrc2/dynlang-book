\documentclass[../index.tex]{subfiles}
\begin{document}
\section{Что такое динамическая форма?}

\section{Режимы открытия динамических форм}
Динамические формы поддерживают несколько режимов работы:

\begin{enumerate}
	\item \verb|MODE_CREATE_INTERACTION| -- режим создания нового обращения из шаблона
	\item \verb|MODE_OPEN_INTERACTION| -- режим чтения обращения
    \item \verb|MODE_OPEN_INCIDENT| -- режим работы с инцидентом
	\item \verb|MODE_COPY_TEMPLATE| -- режим копирования обращения
\end{enumerate}

\subsection{MODE\_CREATE\_INTERACTION}
В режиме создания обращения возможность редактирования полей устанавливается 
флагом \verb|sbmodify| в элементах шаблона: если он равен \verb|true| -- то динамическое поле доступно для редактирования, если такую возможность подразумевает типа данного поля.
При открытии шаблона выполняются динамические команды, указанные в свойствах  \verb|sbcommand| динамических
полей последовательно, в порядке их объявления в шаблоне. Кнопка <<Создать обращение>> отображается по умолчанию
	
\subsection{MODE\_OPEN\_INTERACTION}
В режиме чтения обращения все поля по умолчанию закрыты для редактирования, флаг \verb|sbmodify| игнорируется.
Для <<Разблокирования>> некоторых полей существует механизм \verb|unreadEditableFields| -- данный атрибут обращения соддержит массив идентификаторов динамических полей, которые будут доступны для редактирования в данном режиме. Команды \verb|sbcommand| выполняются только в разблокированных полях, вместо них отрабатывают команды команды из атрибутов \verb|sbonload| динамических полей. Кнопка <<Сохранить обращение>> отображается только в том случае, если массив \verb|unreadEditableFields| непустой.

\subsection{MODE\_OPEN\_INCIDENT}
Режим работы с инцидентом аналогичен с режиму чтения обращения -- все поля закрыты для редактирования, однако, механизм \verb|unreadEditableFields| недоступен. При этом в интерфейсе доступны дополнительные действия для управления инцидентом: <<Решить>>,  <<Назначить на другую группу>> и <<Вернуть диспетчеру>>
Поддерживается только в среде выполнения dom-core.

\subsection{MODE\_COPY\_TEMPLATE}
Режим копирования обращения.

\section{Текстовые поля}

\section{Списки}

\section{Чекбоксы}

\section{}

\end{document}