\documentclass[../index.tex]{subfiles}
\begin{document}
\section{Что такое динамическая форма?}

\section{Режимы открытия динамических форм}
Динамические формы поддерживают несколько режимов работы:

\begin{enumerate}
	\item \verb|MODE_CREATE_INTERACTION| -- режим создания нового обращения из шаблона
	\item \verb|MODE_OPEN_INTERACTION| -- режим чтения обращения
    \item \verb|MODE_OPEN_INCIDENT| -- режим работы с инцидентом
	\item \verb|MODE_COPY_TEMPLATE| -- режим копирования обращения
\end{enumerate}

\subsection{MODE\_CREATE\_INTERACTION}
В режиме создания обращения возможность редактирования полей устанавливается 
флагом \verb|sbmodify| в элементах шаблона: если он равен \verb|true| -- то динамическое поле доступно для редактирования, если такую возможность подразумевает типа данного поля.
При открытии шаблона выполняются динамические команды, указанные в свойствах  \verb|sbcommand| динамических
полей последовательно, в порядке их объявления в шаблоне. Кнопка <<Создать обращение>> отображается по умолчанию
	
\subsection{MODE\_OPEN\_INTERACTION}
В режиме чтения обращения все поля по умолчанию закрыты для редактирования, флаг \verb|sbmodify| игнорируется.
Для <<Разблокирования>> некоторых полей существует механизм \verb|unreadEditableFields| -- данный атрибут обращения соддержит массив идентификаторов динамических полей, которые будут доступны для редактирования в данном режиме. Команды \verb|sbcommand| выполняются только в разблокированных полях, вместо них отрабатывают команды команды из атрибутов \verb|sbonload| динамических полей. Кнопка <<Сохранить обращение>> отображается только в том случае, если массив \verb|unreadEditableFields| непустой.

\subsection{MODE\_OPEN\_INCIDENT}
Режим работы с инцидентом аналогичен с режиму чтения обращения -- все поля закрыты для редактирования, однако, механизм \verb|unreadEditableFields| недоступен. При этом в интерфейсе доступны дополнительные действия для управления инцидентом: <<Решить>>,  <<Назначить на другую группу>> и <<Вернуть диспетчеру>>
Поддерживается только в среде выполнения dom-core.

\subsection{MODE\_COPY\_TEMPLATE}
Режим копирования обращения.
\section{Общее описания динамического поля}
	Каждое динамическое поле представляет собой набор стандартных и дополнительных параметров отвечающих за визуальное отображение и функциональное поеведение элемента.
	Определение элемента задается при формировании шаблонов в HPSM, разнообразие типов достигается комбинацией значений параметров \textbf{sbtype}, \textbf{type}, \textbf{logicalType}, \textbf{objectType}, \textbf{logicalType}
	В качестве примера приведем простое текстовое поле:
\begin{verbatim}
	{
		"objectType": "text",
		"logicalType": "text",
		"id": "editmarriagenumauto",
		"label": "Свидетельство о браке",
		"mandatory": false,
		"sbcommand": "",
		"sbmask": "",
		"sbmodify": true,
		"sbtask": "",
		"sbtitle": "например \"VI-МЮ-777777\"",
		"sbtype": "",
		"sbstyle": "width:33%",
		"style": "text",
		"visible": false,
		"width": "",
		"sbobject": null,
		"sbdbfield": null,
		"sbaction": null,
		"sbmode": null,
		"sbonload": null,
		"sbcopyinfo": false,
		"childs": null,
		"groupid": "",
		"text": null,
		"options": null,
		"type": "2",
		"multiline": null,
		"button": null,
		"matchTable": null,
		"matchField": null,
		"query": null,
		"hpcGroupByFields": null,
		"sbfield": ""
	}
\end{verbatim}
	и элемент типа select:
\begin{verbatim}
	{
		"objectType": "select",
		"logicalType": "combo",
		"id": "purpose",
		"label": "Цель поездки",
		"mandatory": true,
		"sbcommand": "setvalue([something],[purpose])",
		"sbmask": "",
		"sbmodify": true,
		"sbtask": "setvalue([something],[purpose])",
		"sbtitle": "",
		"sbtype": "",
		"sbstyle": "",
		"style": "combo",
		"visible": true,
		"width": "",
		"sbobject": null,
		"sbdbfield": "purpose",
		"sbaction": null,
		"sbmode": null,
		"sbonload": "setvalue([something],[purpose])",
		"sbcopyinfo": false,
		"children": null,
		"groupid": "",
		"text": "Встречи с клиентами",
		"options": [
			{
			"text": "Выезды на аварии",
			"label": "id02",
			"id": "id02"
			},
			{
			"text": "Встречи с клиентами",
			"label": "Встречи с клиентами что-т там",
			"id": "id03"
			},
			{
			"text": "Выезды в государственные органы",
			"label": "Выезды в государственные органы",
			"id": "id04"
			}
		],
		"type": "2",
		"multiline": null,
		"button": null,
		"matchTable": null,
		"matchField": null,
		"query": null,
		"hpcGroupByFields": null,
		"sbfield": ""
	}
\end{verbatim}
\subsection{Параметры динамического поля}
\begin{enumerate}
	\item \verb|objectType| - параметр отвечающий за определение типа поля
	\item \verb|logicalType| - параметр отвечающий за определение типа поля
	\item \verb|sbtype| - параметр отвечающий за определение типа поля
	\item \verb|type| - параметр отвечающий за определение типа поля
	\item \verb|style| - параметр отвечающий за определение типа поля
	\item \verb|id| - уникальный идентификатор поля используется в командах динамического языка для поиска элемента
	\item \verb|text| - значение поля по умолчанию
	\item \verb|label| - заголовок поля, отображаемое значение для информации и обозначения поля
	\item \verb|sbtitle| - подсказка для поля, содержит дополнительную информацию о поле, выводиться в виде вопроса
	\item \verb|groupid| - параметр сортировки и формирования для плоских и legacy групп
	\item \verb|sbcommand| - содержит команды динамического языка выполняемые при инициализации \verb|MODE_CREATE_INTERACTION|
	\item \verb|sbtask| - содержит команды динамического языка выполняемые при изменении поля через пользовательский ввода
	\item \verb|sbonload| - содержит команды динамического языка выполняемые при инициализации \verb|MODE_OPEN_INTERACTION| или \verb|MODE_OPEN_INCIDENT|
	\item \verb|mandatory| - флаг отвечающий за обязательность поля
	\item \verb|sbmask| - маска поля ввода(используется для текстовых полей и дат)
	\item \verb|sbmodify| - флаг определяющий можно ли изменять поле
	\item \verb|sbstyle| - параметр отвечающий за визуальные стили элемента вызывает команду setStyle над элементом
	\item \verb|visible| - флаг определяющий видимость поля в пользовательском интерфейсе
	\item \verb|width| - значение в \% ширины элемента относительно блока
	\item \verb|sbobject| - HP SM property, не изменяется в рамках обработки
	\item \verb|sbdbfield| - HP SM property, не изменяется в рамках обработки
	\item \verb|sbaction| - HP SM property, может быть изменено командой setAction
	\item \verb|sbmode| - HP SM property, может быть изменено командой setMode
	\item \verb|sbcopyinfo| - флаг отвечающий за использовани
	\item \verb|children| - дочерние элементы используется в элементе типа group
	\item \verb|options| - варианты значений для поля для полей с выбором значений
	\item \verb|multiline| - флаг отвечающий за возможность многострочного ввода для текстовых полей
	\item \verb|button| - HP SM property, не изменяется в рамках обработки
	\item \verb|matchTable| - HP SM property, не изменяется в рамках обработки
	\item \verb|matchField| - HP SM property, не изменяется в рамках обработки
	\item \verb|query| - HP SM property, не изменяется в рамках обработки
	\item \verb|hpcGroupByFields| - HP SM property, не изменяется в рамках обработки
	\item \verb|sbfield| - HP SM property, не изменяется в рамках обработки
\end{enumerate}

\subsection{Управление состоянием}

\section{Текстовые поля}
	Текстовые поля могут быть четырех разных визуальных типов + все поля для которых не определены правила тоже преобразуются в обычное текстовое поле
	\begin{enumerate}
		\item Обычное текстовое поле
		\item Неизменяемое текстовое поле(Label) - текстовое поле с флагом \textbf{modify} = "false"
		\item Многострочное текстовое поле - текстовое поле с флагом \textbf{multiline} = "true" и параметром \textbf{type} = "2"
		\item Числовое поле - текстовое поле с параметром \textbf{type} = "1" или \textbf{sbtype} = \verb|NUMBER|
	\end{enumerate}
	Текстовые поля поддерживают ввод по маске(параметр \textbf{mask} не пустой)

	\subsection{Маски ввода}
		Для ограничения ввода в текстовые поля используются обобщенные маски. Маски представляют собой собственные элементы по регулярным выражениям:
		\begin{enumerate}
			\item[\#] - число аналог стандартного [0-9]
			\item[9] - число или пробел [0-9 ]
			\item[A] - символ в верхнем регистре [A-ZА-Я]
			\item[a] - символ в нижнем регистре [a-zа-я]
			\item[B] - символ в любом регистре [A-ZА-Яa-zа-я]
			\item[C] - символ в люом регистре или число [A-ZА-Яa-zа-я ]
		\end{enumerate}
		Остальные символы не из набра масок соответствует статическим символам, которые не участвуют в валидации маски
\section{Поля с выбором значения}
	Динамическая форма поддерживает различные комбинации элементов с возможность выбрать одно из предлагаемых значений, ниже мы рассмотрим подробнее каждый из элементов
\subsection{Select}
	Есть два варианта определния стандартного поле с выбором значения, которые определяются по одному из параметров \textbf{sbtype} = \verb|COMBOAREA| или \textbf{style} = \verb|COMBO|
	Варианты выбора могут быть записаны в поле в параметре \textbf{options}, а так же заполняться из команд динамического языка. Вариантами выбора могут быть текстовые значения, ссылки, изображения, а также их комбинация.
\subsection{Suggest}
	Suggest или поле с подсказкой - расширение стандартного поля Select с возможностью поиска и фильтрации.
	Элемент определяется параметром \textbf{sbtype} = \verb|SUGGEST|
	Частными случаями поля типа Suggest являются поля со значениями \textbf{sbtype} = \verb|SEARCH| и \verb|ADDRESS| предназначенные для поиска(запросы к Автоисполнятору) и получение адреса по вводу города/улицы/дома
	Для поля suggest есть дополнительный параметр \verb|longSuggestItems| - отвечает за визуальное отображение элементов с длинными названиями в вариантах.
\subsection{Чекбоксы}
	Чекбокс - элементы позволяющие сделать выбор из вариантов Да/Нет, Выбрано/Не выбрано.
	Определяется по параметрам \textbf{style} = \verb|CHECKBOX| или \textbf{objectType} = \verb|CHECKBOX|
\subsection{Радиокнопки}
	Элемент типа радиокнопка определяется параметром \textbf{style} = \verb|RADIO| и представляет собой выбор одно из предлагаемых значение.
	Элемент логически не отличается от стандартного Select, но визуальное и функциональное поведение в рамках выполнения команд динамического языка отличается(подробнее в \autoref{sec:dynfom:selectspecial})
\subsection{Мультиэлементы}
	Все типы полей кроме радиокнопок имею свое представление в виде полей со множественным выбором -  \verb|multiSelect|, \verb|multiSuggest|, \verb|multiAddress|, \verb|multiCheckbox|.
	Все мультиэлементы определяются параметром \textbf{sbtype} соответствующим типу поля. Визуально \verb|multiSelect| и \verb|multiSuggest| аналогичны своим единичным типам, но добавляют возможность добавлять несколько значений.
	При выборе значения сохраняются в поле \textbf{text} через символ ";". Так как не все системы поддерживают элементы со множественным выбором - дополнительно придуман метод сохранения таких полей, когда исходное поле визуально скрывается, а после него вставляется текстовое поле с многострочным вводом, каждая строка которого соответствует выбранному значению в исходном поле с индексом значения в начале строки.
\subsection{Осбенности элементов с выбором значения}\label{sec:dynfom:selectspecial} 
	Все элементы с выбором значения кроме радиокнопок, чекбоксов и мультичекбоксов имею важную особенность при выполнении команд динамического языка: Если командой динамического языка вызвать установку значения элемента через функции \verb|setValue|, \verb|setValues| или \verb|setSelectedValue| при этом передать одно единственное значение, то у элемента происходит запуск \verb|sbtask| таким образом эмулируя действия пользователя.
	Примечание: функция \verb|setSelectedValue| всегда вызывает запуск \verb|sbtask| не зависимо от количества переданных значений для установки.
\section{}

\end{document}