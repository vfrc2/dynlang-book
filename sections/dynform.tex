\documentclass[../index.tex]{subfiles}
\begin{document}
\section{Что такое динамическая форма?}
\emergencystretch 3em
Динамическая форма состоит из набора динамических полей, представленных в древовидной структуре.
Поля представляют собой стандартные элементы UI: поля ввода текста, чекбоксы, раскрывающиеся списки, кнопки и т. д.
Они взаимодействуют между собой, с пользователем и внешними системами,
что в конечном счете позволяет автоматизировать любой бизнес процесс.

На низком уровне динамическая форма описана следуя синтаксису XML.
Рассмотрим пример простейшего шаблона динамической формы:
\begin{verbatim}
    <form>
        <text id="helloworld"
              label="Простое текстовое поле"
              mandatory="false"
              sbcommand=""
              sbcopyinfo="false"
              sbfield=""
              sbmask=""
              sbmodify="true"
              sbstyle=""
              sbtask=""
              sbtitle=""
              sbtype=""q
              setGroupId=""
              style="text"
              type="2"
              visible="true" width=""></text>
    </form>
\end{verbatim}

В пользовательском интерфейсе портала «Лицо ДРУГа», динамическая форма будет выглядеть так:

\begin{figure}[h]
    \includegraphics[width=1.0\textwidth]{simpleDynamicForm}
    \centering
\end{figure}

\textbf{Примечание}

Заметьте, что на изображении присутствуют элементы интерфейса «Добавить файл» и «Отправить».
Они не описаны в динамической форме и являются стандартными для всех шаблонов.
Но все же поля динамической формы имеют возможность управлять состоянием этих «статических» элементов
при помощи функций динамического языка.

\section{Конструктор динамических форм}
Во главе концепций динамических форм ставилась скорость и легкость их разработки/сопровождения.
Но написание даже самой простой динамической формы в XML – непростая задача.
Необходимо определить множество свойств, полей, событий и отслеживать взаимосвязи между ними.

Для упрощения процесса создания динамической формы был разработан удобный графический конструктор
динамических форм на платформе HP Service Manager:

\begin{figure}[h]
    \includegraphics[width=1.0\textwidth]{dynamicFormConstructor}
    \centering
\end{figure}

Описание конструктора динамических форм, выходит за рамки данной книги.
Его интерфейс интуитивно понятен, поэтому не вызовет трудностей в освоении,
даже для людей без опыта в разработке UI.

\section{Режимы открытия динамических форм}
Динамические формы поддерживают несколько режимов работы:

\begin{enumerate}
    \item \verb|MODE_CREATE_INTERACTION| -- режим создания нового обращения из шаблона
    \item \verb|MODE_OPEN_INTERACTION| -- режим чтения обращения
    \item \verb|MODE_OPEN_INCIDENT| -- режим работы с инцидентом
    \item \verb|MODE_COPY_TEMPLATE| -- режим копирования обращения
\end{enumerate}

\subsection{MODE\_CREATE\_INTERACTION}
В режиме создания обращения возможность редактирования полей устанавливается 
флагом \verb|sbmodify| в элементах шаблона: если он равен \verb|true| -- то динамическое поле доступно для редактирования, если такую возможность подразумевает типа данного поля.
При открытии шаблона выполняются динамические команды, указанные в свойствах  \verb|sbcommand| динамических
полей последовательно, в порядке их объявления в шаблоне. Кнопка <<Создать обращение>> отображается по умолчанию
    
\subsection{MODE\_OPEN\_INTERACTION}
В режиме чтения обращения все поля по умолчанию закрыты для редактирования, флаг \verb|sbmodify| игнорируется.
Для <<Разблокирования>> некоторых полей существует механизм \verb|unreadEditableFields| -- данный атрибут обращения содержит массив идентификаторов динамических полей, которые будут доступны для редактирования в данном режиме. Команды \verb|sbcommand| выполняются только в разблокированных полях, вместо них отрабатывают команды из атрибутов \verb|sbonload| динамических полей. Кнопка <<Сохранить обращение>> отображается только в том случае, если массив \verb|unreadEditableFields| непустой.

\section{Наследование состояния\\динамической формы}
Ранее упоминалось, что динамическая форма после исполнения и преобразования пользователем,
сохраняет свое состояние в БД. Из этого состояния может быть создана «производная» динамическая форма,
которая будет наследовать некоторые свойства исходной. Эта возможность бывает очень полезной,
когда мы хотим освободить пользователя от необходимости повторно заполнять одинаковые динамические поля.

Под наследованием состояния динамической формы, понимается последовательный перенос свойств динамических полей
из исходной формы в производную. Перенос производится для динамических полей, совпадающих по свойству \verb|id|.


Наследуются следующие свойства:
\begin{itemize}
    \item \verb|value|
    \item \verb|visible|
    \item \verb|enabled|
    \item \verb|sbmodify|
\end{itemize}
\vspace{5mm}


Некоторые типы динамических элементов, содержат значения в виде набора дочерних элементов,
например: чек-боксы, радио-боксы, выпадающие списки. Эти дочерние элементы имеют свои
свойства: видимость, доступность для редактирования и значение.
Свойства дочерних элементов тоже должны быть унаследованы, и передаются в
атрибутах:
\begin{itemize}
  \item \verb|valueSelected|
  \item \verb|valueEnabled|
  \item \verb|valueVisible|
\end{itemize}
\vspace{5mm}


\textbf{Примечание}


У динамических полей, которые наследовали свойства,
не будут работать команды динамического языка при старте (\verb|sbcommand|).
Обычно команды при старте используются для инициализации значений в динамических полях.
Поэтому для избегания конфликтов с наследованием, эти команды отключены.


Существует три способа наследования состояния:


\begin{enumerate}
    \item копирование динамической формы (кнопка «Копировать»);
    \item создание связанного обращения;
    \item передача свойств динамической формы, через ссылку в аргументах URL.
\end{enumerate}


\subsection{Кнопка «Копировать»}
Бизнес услуги могут быть востребованы пользователем несколько раз за короткий промежуток времени.
Пользователь, заполняет форму (обращения) получения услуги один раз, а далее создает копию этого обращения
и лишь незначительно меняет его поля. Для этого, в режиме просмотра обращения (по некоторым услугам),
доступна кнопка «Копировать».

Нажатие кнопки «Копировать», переадресовывает пользователя
на страницу создания нового обращения по услуге исходного.
Свойства динамических полей будут унаследованы по полю \verb|id|, как было указано ранее.


\vspace{5mm}
\textbf{Примечание}


Копирование динамических полей, может отрабатывать неочевидным образом,
а в некоторых формах вовсе приводит к некорректным заявкам:


\begin{itemize}
  \item При наследовании значения у полей типа «Дата», будет наследована дата,
  которую пользователь указал несколько дней назад.
  При копировании пользователь может не учесть этого и не скорректировать дату с учетом прошедшего времени.
  \item В \verb|sbcommand| могут быть использованы команды инициализации значения из внешней системы,
  например из автоисполнятора. Часто параметры во внешних системах изменяются и замена
  такой инициализации на наследование устанавливает неактуальные значения.
\end{itemize}


В таких динамических формах необходимо отключать кнопку «Копировать».
Отключить её можно при помощи функции динамического языка:


\begin{verbatim}
    setCopyVisible([bool],[id])
\end{verbatim}

\subsection{Создание связанного обращения}
Исполнение обращения пользователя может потребовать запуск дополнительных бизнес процессов.
Их можно инициировать, создав связанное обращение к исходному.
Такое связанное обращение будет наследовать и хранить информацию об исходном (родительском) обращении.


Для создания связанного обращения используется специальная ссылка, перейдя по которой,
пользователь попадает на страницу создания связанного обращения.
Для портала «Лицо ДРУГа», ссылка имеет следующую структуру:


\begin{verbatim}
    https://friend.sbrf.ru/FriendFace/#createInteraction?FRIEND
        &Обеспечение_IT
        &SD227598282
        &INTERACTION
        &PARENT
        &Филатов Валентин Васильевич/
\end{verbatim}


В параметрах URL передается:
\begin{itemize}
    \item \verb|FRIEND| – подсистема, в которую будет направлено обращение;
    \item \verb|Обеспечение_IT| – идентификатор шаблона;
    \item \verb|SD227598282| – идентификатор исходного обращения;
    \item \verb|INTERACTION| – тип исходного обращения;
    \item \verb|PARENT| – тип связи с исходным обращением;
    \item \verb|Филатов Валентин Васильевич| – идентификатор создателя исходного обращения.
\end{itemize}


Связанная (производная) динамическая форма будет наследовать состояние исходной по элементам с
совпадающими полями \verb|id|, как было описано ранее в этом разделе.


Особенностью данного способа наследования является то,
что шаблон производной динамической формы не совпадает с шаблоном исходной.
Разные шаблоны имеют разные наборы динамических элементов,
поэтому одного наследования по \verb|id| в данном способе недостаточно.
Динамические элементы, которых нет в производной форме,
будут полностью скопированы в отдельную группу элементов и помещены в конец производной формы.

\begin{figure}[h]
    \includegraphics[width=0.8\textwidth]{relatedInteraction}
    \centering
    \caption{Пример формы наследования связанного обращения}
\end{figure}

Динамическое поле «counterAlert» – часть исходной динамической формы,
а поля «Наименование соглашения», «Действие», «Договор/объемы» – наследованные динамические элементы,
которые объединены в группу «Информация обращения SD227598282».
Обратите внимание, все элементы наследованной группы недоступны к изменению.


\section{Общее описание динамического поля}
    Каждое динамическое поле представляет собой набор стандартных и дополнительных параметров отвечающих за визуальное отображение и функциональное поведение элемента.
    Определение элемента задается при формировании шаблонов в HPSM, разнообразие типов достигается комбинацией значений параметров \textbf{sbtype}, \textbf{type}, \textbf{logicalType}, \textbf{objectType}, \textbf{logicalType}
    В качестве примера приведем простое текстовое поле:
\begin{verbatim}
    {
        "objectType": "text",
        "logicalType": "text",
        "id": "editmarriagenumauto",
        "label": "Свидетельство о браке",
        "mandatory": false,
        "sbcommand": "",
        "sbmask": "",
        "sbmodify": true,
        "sbtask": "",
        "sbtitle": "например \"VI-МЮ-777777\"",
        "sbtype": "",
        "sbstyle": "width:33%",
        "style": "text",
        "visible": false,
        "width": "",
        "sbobject": null,
        "sbdbfield": null,
        "sbaction": null,
        "sbmode": null,
        "sbonload": null,
        "sbcopyinfo": false,
        "childs": null,
        "groupid": "",
        "text": null,
        "options": null,
        "type": "2",
        "multiline": null,
        "button": null,
        "matchTable": null,
        "matchField": null,
        "query": null,
        "hpcGroupByFields": null,
        "sbfield": ""
    }
\end{verbatim}
    и элемент типа select:
\begin{verbatim}
    {
        "objectType": "select",
        "logicalType": "combo",
        "id": "purpose",
        "label": "Цель поездки",
        "mandatory": true,
        "sbcommand": "setvalue([something],[purpose])",
        "sbmask": "",
        "sbmodify": true,
        "sbtask": "setvalue([something],[purpose])",
        "sbtitle": "",
        "sbtype": "",
        "sbstyle": "",
        "style": "combo",
        "visible": true,
        "width": "",
        "sbobject": null,
        "sbdbfield": "purpose",
        "sbaction": null,
        "sbmode": null,
        "sbonload": "setvalue([something],[purpose])",
        "sbcopyinfo": false,
        "children": null,
        "groupid": "",
        "text": "Встречи с клиентами",
        "options": [
            {
            "text": "Выезды на аварии",
            "label": "id02",
            "id": "id02"
            },
            {
            "text": "Встречи с клиентами",
            "label": "Встречи с клиентами что-т там",
            "id": "id03"
            },
            {
            "text": "Выезды в государственные органы",
            "label": "Выезды в государственные органы",
            "id": "id04"
            }
        ],
        "type": "2",
        "multiline": null,
        "button": null,
        "matchTable": null,
        "matchField": null,
        "query": null,
        "hpcGroupByFields": null,
        "sbfield": ""
    }
\end{verbatim}
\subsection{Параметры динамического поля}
\begin{itemize}
    \item \verb|objectType| -- параметр отвечающий за определение типа поля
    \item \verb|logicalType| -- параметр отвечающий за определение типа поля
    \item \verb|sbtype| -- параметр отвечающий за определение типа поля
    \item \verb|type| -- параметр отвечающий за определение типа поля
    \item \verb|style| -- параметр отвечающий за определение типа поля
    \item \verb|id| -- уникальный идентификатор поля используется в командах динамического языка для поиска элемента
    \item \verb|text| -- значение поля по умолчанию
    \item \verb|label| -- заголовок поля, отображаемое значение для информации и обозначения поля
    \item \verb|sbtitle| -- подсказка для поля, содержит дополнительную информацию о поле, выводиться в виде вопроса
    \item \verb|groupid| -- параметр сортировки и формирования для плоских и legacy групп
    \item \verb|sbcommand| -- содержит команды динамического языка выполняемые при инициализации \verb|MODE_CREATE_INTERACTION|
    \item \verb|sbtask| -- содержит команды динамического языка выполняемые при изменении поля через пользовательский ввода
    \item \verb|sbonload| -- содержит команды динамического языка выполняемые при инициализации \verb|MODE_OPEN_INTERACTION| или \verb|MODE_OPEN_INCIDENT|
    \item \verb|mandatory| -- флаг отвечающий за обязательность поля
    \item \verb|sbmask| -- маска поля ввода(используется для текстовых полей и дат)
    \item \verb|sbmodify| -- флаг определяющий можно ли изменять поле
    \item \verb|sbstyle| -- параметр отвечающий за визуальные стили элемента вызывает команду setStyle над элементом
    \item \verb|visible| -- флаг определяющий видимость поля в пользовательском интерфейсе
    \item \verb|width| -- значение в \% ширины элемента относительно блока
    \item \verb|sbobject| -- HP SM property, не изменяется в рамках обработки
    \item \verb|sbdbfield| -- HP SM property, не изменяется в рамках обработки
    \item \verb|sbaction| -- HP SM property, может быть изменено командой setAction
    \item \verb|sbmode| -- HP SM property, может быть изменено командой setMode
    \item \verb|sbcopyinfo| -- флаг отвечающий за копирование значения динамического поля в физическое поле information
    \item \verb|children| -- дочерние элементы используется в элементе типа group
    \item \verb|options| -- варианты значений для поля для полей с выбором значений
    \item \verb|multiline| -- флаг отвечающий за возможность многострочного ввода для текстовых полей
    \item \verb|button| -- HP SM property, не изменяется в рамках обработки
    \item \verb|matchTable| -- HP SM property, не изменяется в рамках обработки
    \item \verb|matchField| -- HP SM property, не изменяется в рамках обработки
    \item \verb|query| -- HP SM property, не изменяется в рамках обработки
    \item \verb|hpcGroupByFields| -- HP SM property, не изменяется в рамках обработки
    \item \verb|sbfield| -- HP SM property, не изменяется в рамках обработки
\end{itemize}
>>>>>>> small edits

\subsection{Управление состоянием}

\section{Текстовые поля}
    Текстовые поля могут быть четырех разных визуальных типов + все поля для которых не определены правила тоже преобразуются в обычное текстовое поле
    \begin{enumerate}
        \item Обычное текстовое поле
        \item Неизменяемое текстовое поле(Label) - текстовое поле с флагом \textbf{modify} = "false"
        \item Многострочное текстовое поле - текстовое поле с флагом \textbf{multiline} = "true" и параметром \textbf{type} = "2"
        \item Числовое поле - текстовое поле с параметром \textbf{type} = "1" или \textbf{sbtype} = \verb|NUMBER|
    \end{enumerate}
    Текстовые поля поддерживают ввод по маске(параметр \textbf{mask} не пустой)

    \subsection{Маски ввода}
        Для ограничения ввода в текстовые поля используются обобщенные маски. Маски представляют собой собственные элементы по регулярным выражениям:
        \begin{enumerate}
            \item[\#] - число аналог стандартного [0-9]
            \item[9] - число или пробел [0-9 ]
            \item[A] - символ в верхнем регистре [A-ZА-Я]
            \item[a] - символ в нижнем регистре [a-zа-я]
            \item[B] - символ в любом регистре [A-ZА-Яa-zа-я]
            \item[C] - символ в любом регистре или число [A-ZА-Яa-zа-я ]
        \end{enumerate}
        Остальные символы не из набора масок соответствует статическим символам, которые не участвуют в валидации маски
\section{Поля с выбором значения}
    Динамическая форма поддерживает различные комбинации элементов с возможность выбрать одно из предлагаемых значений, ниже мы рассмотрим подробнее каждый из элементов
\subsection{Select}
    Есть два варианта определения стандартного поле с выбором значения, которые определяются по одному из параметров \textbf{sbtype} = \verb|COMBOAREA| или \textbf{style} = \verb|COMBO|
    Варианты выбора могут быть записаны в поле в параметре \textbf{options}, а так же заполняться из команд динамического языка. Вариантами выбора могут быть текстовые значения, ссылки, изображения, а также их комбинация.
\subsection{Suggest}\label{suggest}
    Suggest или поле с подсказкой - расширение стандартного поля Select с возможностью поиска и фильтрации.
    Элемент определяется параметром \textbf{sbtype} = \verb|SUGGEST|
    Частными случаями поля типа Suggest являются поля со значениями \textbf{sbtype} = \verb|SEARCH| и \verb|ADDRESS| предназначенные для поиска(запросы к Автоисполнятору) и получение адреса по вводу города/улицы/дома
    Для поля suggest есть дополнительный параметр \verb|longSuggestItems| - отвечает за визуальное отображение элементов с длинными названиями в вариантах.
\subsection{Чекбоксы}
    Чекбокс - элементы позволяющие сделать выбор из вариантов Да/Нет, Выбрано/Не выбрано.
    Определяется по параметрам \textbf{style} = \verb|CHECKBOX| или \textbf{objectType} = \verb|CHECKBOX|
\subsection{Радиокнопки}
    Элемент типа радиокнопка определяется параметром \textbf{style} = \verb|RADIO| и представляет собой выбор одно из предлагаемых значение.
    Элемент логически не отличается от стандартного Select, но визуальное и функциональное поведение в рамках выполнения команд динамического языка отличается(подробнее в \autoref{sec:dynfom:selectspecial})
\subsection{Мультиэлементы}
    Все типы полей кроме радиокнопок имею свое представление в виде полей со множественным выбором -  \verb|multiSelect|, \verb|multiSuggest|, \verb|multiAddress|, \verb|multiCheckbox|.
    Все мультиэлементы определяются параметром \textbf{sbtype} соответствующим типу поля. Визуально \verb|multiSelect| и \verb|multiSuggest| аналогичны своим единичным типам, но добавляют возможность добавлять несколько значений.
    При выборе значения сохраняются в поле \textbf{text} через символ ";". Так как не все системы поддерживают элементы со множественным выбором - дополнительно придуман метод сохранения таких полей, когда исходное поле визуально скрывается, а после него вставляется текстовое поле с многострочным вводом, каждая строка которого соответствует выбранному значению в исходном поле с индексом значения в начале строки.
\subsection{Особенности элементов с выбором значения}\label{sec:dynfom:selectspecial} 
    Все элементы с выбором значения кроме радиокнопок, чекбоксов и мультичекбоксов имею важную особенность при выполнении команд динамического языка: Если командой динамического языка вызвать установку значения элемента через функции \verb|setValue|, \verb|setValues| или \verb|setSelectedValue| при этом передать одно единственное значение, то у элемента происходит запуск \verb|sbtask| таким образом эмулируя действия пользователя.
    Примечание: функция \verb|setSelectedValue| всегда вызывает запуск \verb|sbtask| не зависимо от количества переданных значений для установки.

\section{Окружение}  %TODO придумать норм название

Динамический язык выполняется в определенном окружении. Окружение позволяет получить доступ к данным пользователя,
аттрибутам шаблона и т.д. Доступ к этим данным осуществляется через команды дин языка. 
Окружение полезно для автоматического заполнения шаблона данными о пользователе, изменить поведение созданного обращения,
например запросить подпись с помощью механизма ActiveX.

При создании обращения, в зависимости от шаблона могут потребоваться дополнительные данные или определенный порядок 
создания обращения, для этого существую специальные флаги которые относятся ко всей форме сразу.

\subsection{Поле информация}
У каждого обращение существует поле информация, данное поле необходимо для правильной работы HP Service Manager.
Некоторые шаблоны не используют данное поле и его можно скрыть используя флаг $informationVisible$ и команду
\verb|setInformationVisible([bool], [id])|, также можно установить значение этого поля с помощью:
\begin{verbatim}
setDescription([string], [id])
\end{verbatim}

\subsection{Прикрепление файлов}
В некоторых шаблонах необходимо, чтобы пользователь мог отправить файл, который называется вложением и связать его с сущностью обращения.
Для этого используют флаги $fileEnable$ и $fileMandatory$, для запрета вложений и для обязательности вложения файла соответственно.
Для чтения и установки флагов используются команды:
\begin{verbatim}
    isFileEnabled()
    isFileMandatory()
    setFileEnabled([bool],[id])
    setFileMandatory([bool], [id])
\end{verbatim}

Флаг $fileEnable$ запрещает прикладывать файлы для всей формы, но если в дин форме есть отдельный компонент вложения, то
файл все равно можно будет приложить.

\subsection{Внутренний клиент}

У каждого обращения есть внутренний клиент (ВК), это сотрудник которому необходимо оказать услугу. В большинстве случаев 
ВК это тот пользователь который вошел в приложение. Но бывают случаи когда его необходимо сменить, например если сломался компьютер,
и необходимо создать обращение на его пос=чинку, то можно попросить коллегу создать обращение от твоего имени.

Для управления полем ВК используются команды:

\verb|setInitiatorVisible([bool], [id])| -- позволяет отключить возможность смены ВК, в интерфейсе


\verb|setInitiator([vkId], [id])| -- устанавливает ВК

\subsection{Подписание обращения}
Для включения подписи обращения при сохранении можно использовать команду \verb|setCheckSign|

%TODO Добавить описание процесса подписания и новый формат

\subsection{Связанные обращения}
HP SM позволяет связать обращения между собой. Это необходимо для автоматического изменения статусов и для
организации бизнес процессов. Связь это два поля: id объекта и тип связи. Из динамического языка устанавливаются командами
\begin{verbatim}
    getRelatedId([id])
    setRelationship([relation])
\end{verbatim} 

\subsection{Scoring}
Некоторые шаблоны поддерживаю специальные поля, которые помогают в распределении обращений и их приоритизации.
Эти специальные поля -- отдельная динамическая форма, которая автоматически заполняется из настроек, но в некоторых
случаях пользователь сам может выбрать важность обращения. Для управления видимостью дополнительных полей нужно использовать
\verb|setScoringVisible|

\subsection{Поточный ввод}
Иногда есть необходимость а поточном создании обращений, когда необходимо создать несколько обращений по одному 
шаблону с незначительными изменениями значений полей, для этого существует флаг $MultiFlow$. В это режиме
при отправке обращения не происходит переход на главный экран, а форма остается разрешается изменить поля и 
создать еще одно обращение. Управление этим флагом осуществляется с помощью функции \verb|setMultiFlow([bool])|.

\subsection{Остальные флаги}
Также существую дополнительные флаги которыми можно управлять отображением обращения и доп информацией для HP setMode

\verb|setDeclineVisible([bool],[id])| используется для отображения кнопки "Отозвать" в обращении
\verb|setAdditionalTemplate([templateid])| позволяет задать дополнительный шаблон при создании обращения
\verb|setContainCustomObjects([])| зарезервировано для внутреннего использования

\section{Общие команды динамической формы}
Динамическая форма позволяет выполнить команды дин языка не только при изменении компонентов, но и при открытии формы,
и перед сохранением. Эти команды используются для дополнительной проверки введенных значений, или для скрытия 
технических полей пред открытием уже созданного обращения.

Для установки команды перед сохранением используется
\begin{verbatim}
    setBeforeSave([string([dynlang])])
\end{verbatim}
Дополнительный вызов \verb|string()| необходим для экранирования символов динамического языка. Без него строка динамического языка
была бы интерпретирована как команда и вызвана.

Для установки команды перед открытием необходимо использовать вызов АИ в режиме передачи $3DMap$

\end{document}